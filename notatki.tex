\documentclass{article}

\usepackage[utf8]{inputenc}
\usepackage{geometry} \geometry{margin=85pt}
\usepackage[polish]{babel}
\usepackage{polski}
\usepackage[document]{ragged2e}
\usepackage{amsmath}
\usepackage{amsthm}
\usepackage{amssymb}
\usepackage{amsfonts}
\usepackage{mathtools}
\usepackage{fancyhdr}
\usepackage{enumitem}
\usepackage[breakable,skins]{tcolorbox}
\usepackage{kpfonts}
\usepackage[T1]{fontenc}
\usepackage{xcolor}
\usepackage{relsize}


%%%%%%%%%%%%%%%%%%%
%      TODO       %


% Domknięcie domknięcia to po prostu domknięcie
% Monotoniczność domknięcia
% Szeregi o wyrazach dodatnich cz. I
% Inna deginicja punktu skupienia

%%%%%%%%%%
% KOLORY %
%%%%%%%%%%

\definecolor{col1}{HTML}{ffcccc}
\definecolor{col2}{HTML}{ccccff}
\definecolor{col3}{HTML}{00ff00}
\definecolor{col4}{HTML}{8888cc}
\definecolor{darkred}{HTML}{8B0000}

%%%%%%%%%
% RAMKI %
%%%%%%%%%

\newcounter{defi}
\numberwithin{defi}{section}

\newtcolorbox{defr}[2][]{%
  breakable, enhanced, colback=white,colframe=col1,coltitle=black,
  sharp corners,boxrule=1.5pt,
  fonttitle=\bfseries,top=13pt,
  attach boxed title to top left={yshift=-\tcboxedtitleheight/2, xshift=10pt},
  boxed title style={tile,size=small,left=5pt,right=5pt, 
  colback=col1,before upper=\strut},
  title={Definicja \thedefi . #1 #2},
  before title={\refstepcounter{defi}},
}



% \newcounter{twier}
% \numberwithin{twier}{section}

\newtcolorbox[auto counter,number within= section]{twier}[2][]{%
  breakable, enhanced, colback=white,colframe=col2,coltitle=black,
  sharp corners,boxrule=1.5pt,
  fonttitle=\bfseries,top=13pt,
  attach boxed title to top left={yshift=-\tcboxedtitleheight/2, xshift=10pt},
  boxed title style={tile,size=small,left=5pt,right=5pt, 
  colback=col2,before upper=\strut},   
  title={Twierdzenie \thetcbcounter . #2},
  #1,
%   before title={\refstepcounter{twier}},
  }

\newtcolorbox{dow}[1][Dowód.]{% 
  breakable, enhanced, colback=white,colframe=col3,coltitle=black,
  sharp corners,boxrule=1.5pt,
  fonttitle=\bfseries,top=13pt,
  attach boxed title to top left={yshift=-\tcboxedtitleheight/2, xshift=10pt},
  boxed title style={tile,size=small,left=5pt,right=5pt, 
  colback=col3,before upper=\strut},
  title={#1},
}


\newcounter{obser}
\numberwithin{defi}{section}

\newtcolorbox[auto counter,number within= section]{obs}[2][]{%
  breakable, enhanced, colback=white,colframe=col4,coltitle=black,
  sharp corners,boxrule=1.5pt,
  fonttitle=\bfseries,top=13pt,
  attach boxed title to top left={yshift=-\tcboxedtitleheight/2, xshift=10pt},
  boxed title style={tile,size=small,left=5pt,right=5pt, 
  colback=col4,before upper=\strut},
  title={Obserwacja \thetcbcounter . #2},
  #1
}

%%%%%%%%%%%
% KOMENDY %
%%%%%%%%%%%

\newcommand{\R}{\mathbb{R}}
\newcommand{\Rbar}{\overline{\mathbb{R}}}
\newcommand{\N}{\mathbb{N}}
\newcommand{\Z}{\mathbb{Z}}
\newcommand{\Q}{\mathbb{Q}}
\newcommand{\C}{\mathbb{C}}
\newcommand{\oo}{\infty}
\newcommand{\ifff}{\Leftrightarrow}
\newcommand{\imp}{\Rightarrow}
\newcommand{\Tau}{\mathcal{T}}
\newcommand{\Fau}{\mathcal{F}}
\newcommand{\Nau}{\mathcal{N}}
\newcommand{\subotw}{\underset{\clap{\scriptsize otw.}}{\subseteq}}
\newcommand{\subdomk}{\underset{\clap{\scriptsize domk.}}{\subseteq}}
\newcommand{\dg}{^{\circ}}
\newcommand{\se}{\subseteq}
\newcommand{\id}{\text{id}}
\renewcommand{\inf}{\mathop{\mathrm{inf}\vphantom{\mathrm{sup}}}}

\newcommand{\Sd}{\underline{S}}
\newcommand{\Sg}{\overline{S}}


\newcommand{\Cd}{\mkern3mu\underline{\vphantom{\intop}\mkern7mu}\mkern-10mu\int}
\def\Cg{\mathchoice%
{\mkern13mu\overline{\vphantom{\intop}\mkern7mu}\mkern-20mu}%
{\mkern7mu\overline{\vphantom{\intop}\mkern7mu}\mkern-14mu}%
{\mkern7mu\overline{\vphantom{\intop}\mkern7mu}\mkern-14mu}%
{\mkern7mu\overline{\vphantom{\intop}\mkern7mu}\mkern-14mu}%
\int}

\newcommand{\przedz}[2]{[#1 _ {#2}, #1_{#2 - 1}]}

\providecommand{\ol}{\overline}
\providecommand{\ul}{\underline}
\providecommand{\wt}{\widetilde}
\providecommand{\wh}{\widehat}
\providecommand{\eps}{\varepsilon}
\providecommand{\half}{\frac{1}{2}}
\providecommand{\inv}{^{-1}}
\newcommand{\dang}{\measuredangle} %% Directed angle
\providecommand{\CC}{\mathbb C}
\providecommand{\FF}{\mathbb F}
\providecommand{\NN}{\mathbb N}
\providecommand{\QQ}{\mathbb Q}
\providecommand{\RR}{\mathbb R}
\providecommand{\ZZ}{\mathbb Z}
\providecommand{\ts}{\textsuperscript}
\providecommand{\dg}{^\circ}
\providecommand{\ii}{\item}

\renewcommand{\geq}{\geqslant}
\renewcommand{\leq}{\leqslant}






\newcommand{\tb}[1]{\textbf{#1}}
\newcommand{\bsum}[2]{\mathlarger{\sum_{#1}^{#2}}}
\newcommand{\szerI}[1]{\bsum{n=1}{\oo} #1_n}
\newcommand{\ilorazowy}[1]{#1/\!{_R}}
\newcommand{\ciag}[1]{(#1_{n})_{n \in \N}}
\newcommand{\gras}[2]{\lim_{#1 \to \oo} #2_{#1}}
\newcommand{\lgras}[2]{\mathlarger{\lim}_{#1 \to \oo} #2_{#1}}
\newcommand{\ball}[2]{\text{Ball}(#1, \, #2)}
\newcommand{\dball}[2]{\overline{\text{Ball}}(#1, \, #2)}


\DeclarePairedDelimiter\set\{\}
\ExplSyntaxOn
\NewDocumentCommand{\op}{m}
 {
  \langle
  \clist_set:Nn \l_tmpa_clist { #1 }
  \clist_use:Nn \l_tmpa_clist {,\mspace{3mu plus 1mu minus 1mu}\allowbreak}
  \rangle
}
\ExplSyntaxOff







\title{%
Notatki do Analizy I R \\
\large Na podstawie wykładu głoszonego przez prof. Sołtana w 2023 r.}

\author{red. Filip Baciak}
\date{November 2023}

\begin{document}

\maketitle

\newpage

%%pierwszy wyklad - 3 X 2023
\section{Wstęp}
\subsection{Relacje}
\begin{defr}{Relacja}
    Relacją $R$ ze zbioru $A$ do zbioru $B$ nazywamy podzbiór iloczynu kartezjańskiego tych dwu zbiorów:
    \begin{equation}
        R \subseteq A \times B.
    \end{equation}
    Jeśli $(x, y) \in R$ to piszemy $xRy$.
\end{defr}

\paragraph*{Przykłady relacji:}
\begin{itemize}
    \item Relacja równości $R \subseteq A \times A$, zdefioniowana:
          \begin{equation}
              R = \{(a, a) | a \in A\}.
          \end{equation}
    \item Na zbiorze $\N$ mamy relację wewnętrzną (tj. będącą podzbiorem $\N ^2$):
          \begin{equation}
              R = \{(n, m) | n \leqslant m\}.
          \end{equation}
\end{itemize}

\begin{defr}{Relacja równoważności}
    Relacją równoważności nazywamy relację $R \subseteq A \times A$, spełniającą następujące aksjomaty:
    \begin{enumerate}
        \item Zwrotność:
              \begin{equation}
                  \forall_{x \in A}: xRx.
              \end{equation}
        \item Symetryczność:
              \begin{equation}
                  \forall_{x, y \in A}: xRy \implies yRx.
              \end{equation}
        \item Przechodniość:
              \begin{equation}
                  \forall_{x, y, z\in A}: xRy \land yRz \implies xRz.
              \end{equation}
    \end{enumerate}
\end{defr}

Przykładem relacji równoważności jest relacja $R_f$ zadana przez funkcję $f: A \to B$:
\begin{equation}
    x R_f y \iff f(x) = f(y).
\end{equation}

\begin{defr}{Częściowy porządek}
    Częściowym porządkiem na zbiorze $A$ nazywamy relację $R \subseteq A^2$ (którą oznaczamy $\leqslant$ i piszemy $x \leqslant y$ zamiast $ x R y$), jeśli ma następujące cechy:
    \begin{enumerate}
        \item Zwrotność:
              \begin{equation}
                  \forall_{x \in A}: x \leqslant x.
              \end{equation}
        \item Antysymetryczność:
              \begin{equation}
                  \forall_{x, y \in A}: x \leqslant y \land y \leqslant x \implies x = y.
              \end{equation}
        \item Przechodniość:
              \begin{equation}
                  \forall_{x, y, z\in A}: xRy \land yRz \implies xRz.
              \end{equation}
    \end{enumerate}
    Zbiór parę $(A, \leqslant)$ nazywamy zbiorem częsciowo uporządkowanym.
\end{defr}

Na przykład relacja wewnętrzna na zbiorze $\N ^2$ zdefiniowana następująco:
\[ (a, b) \leqslant (a', b') \iff a \leqslant a' \land b \leqslant b', \]
zadaje częsciowy porządek nad $\N ^2$.

\begin{defr}{Porządek liniowy}
    Porządek częściowy $\leqslant$ nad $A$ nazywamy \textbf{liniowym}, jeśli:
    \begin{equation}
        \forall_{x, y \in A}: \quad x \leqslant y \lor x \leqslant y.
    \end{equation}
    Zbiór z określonym porządkiem liniowym nazywamy \textbf{uporządkowanym liniowo}.
    Jeśli $x \leqslant y \land x \neq y$ to piszemy $x < y$.
\end{defr}

Zauważmy, że porządek częściowy - jak sama nazwa wskazuje - niekoniecznie określa relację większości między każdymi dwoma elementami zbioru na którym jest określony. Tę własność ma dopiero porządek liniowy.

\begin{defr}{Ograniczenia}
    Podzbiór $X \subseteq A$ zbioru uporządkowanego liniowo $(A, \leqslant)$ nazywamy \textbf{ograniczonym z góry}, jeśli:
    \begin{equation}
        \exists_{u \in A} \forall_{x \in X} : x \leqslant B.
    \end{equation}
    Podobnie definiujemy \textbf{ograniczenie z dołu}:
    \begin{equation}
        \exists_{l \in A} \forall_{x \in X}: l \leqslant x.
    \end{equation}
    Elementy $u$ i $l$ nazywamy odpowiednio \textbf{ograniczeniem górnym} i \textbf{ograniczeniem dolnym}.
\end{defr}

\begin{defr}{Kresy górne i dolne}
    \textbf{Kresem górnym} podzbioru $X \subseteq A$ uporządkowanego $(A, \leqslant)$ nazwiemy najmniejsze jego ograniczenie górne, to znaczy taką liczbę $b \in A$, że:
    \begin{itemize}
        \item $b$ jest ograniczeniem górnym $X$,
        \item jeśli $l$ jest ograniczeniem górnym $X$, to $b \leqslant l$.
    \end{itemize}
    Podobnie - jako największe ograniczenie dolne - definiujemy \textbf{kres dolny}.
    Kres górny zbioru $X$ oznaczamy $\sup{X}$, a kres dolny $\inf{X}$
\end{defr}
Zauważmy, że w ogólności zbiór nie musi mieć kresu górnego lub dolnego, a jeśli go ma to kres nie musi być elementem tegoż zbioru.

\subsection{Liczby rzeczywiste}
\begin{defr}{$\R$}
    \textbf{Liczbami rzeczywistymi} nazywamy zbiór $\R$ z określonymi działaniami dodawania $+$ i mnożenia $\cdot$, wyróżnionymi, różnymi elementami $0$ i $1$ i określoną relacją porządku liniowego $\leqslant$ - w skrócie $(\R, +, \cdot, 0, 1, \leqslant)$ - taki że:
    \begin{enumerate}
        \item $\R$ jest ciałem, tzn. spełnia:
              \begin{enumerate}
                  \item Zamkniętość dodawania i mnożenia:
                        \begin{equation}
                            \forall_{a, b \in \R}: a+b \in R \land a \cdot b \in \R;
                        \end{equation}

                  \item $0$ jest elementem neutralnym dodawania:
                        \begin{equation}
                            \forall_{a \in \R}: a + 0 = a;
                        \end{equation}

                  \item Istnieją elementy odwrotne względem dodwania:
                        \begin{equation}
                            \forall_{a \in \R} \exists_{-a \in R}: a + (-a) = 0;
                        \end{equation}

                  \item Dodawanie jest łączne:
                        \begin{equation}
                            \forall_{a, b, c \in \R}: (a + b) + c = a + (b + c);
                        \end{equation}

                  \item Dodawanie jest przemienne:
                        \begin{equation}
                            \forall_{a, b \in \R}: a + b = b + a;
                        \end{equation}

                  \item $1$ jest elementem neutralnym mnożenia:
                        \begin{equation}
                            \forall_{a \in \R}: a \cdot 1 = a;
                        \end{equation}

                  \item Mnożenie jest łączne:
                        \begin{equation}
                            \forall_{a, b, c \in \R}: (a \cdot b) \cdot c = a \cdot (b \cdot c);
                        \end{equation}

                  \item Mnożenie jest przemienne:
                        \begin{equation}
                            \forall_{a, b \in \R}: a \cdot b = b \cdot a;
                        \end{equation}

                  \item Istnieją elementy przeciwne względem mnożenia (z wyjątkiem $0$):
                        \begin{equation}
                            \forall_{a \in \R \backslash \{0\}} \exists_{a^{-1}\in R}: a \cdot a^{-1} = 1;
                        \end{equation}

                  \item Dodawanie jest rozdzielne względem mnożenia:
                        \begin{equation}
                            \forall_{a, b, c \in \R}: a \cdot (b + c) = a \cdot b + a \cdot c.
                        \end{equation}

              \end{enumerate}
        \item Porządek liniowy $\leqslant$ spełnia:
              \begin{enumerate}
                  \item Możliwość dodawania "stronami":
                        \begin{equation}
                            \forall_{a, b, t \in R}: a \leqslant b \implies a + t \leqslant b + t;
                        \end{equation}

                  \item Mnożenie dodatnich zachowuje dodatniość
                        \begin{equation}
                            \forall_{a, b\in R}: 0 < a \land 0< b \implies 0 < a \cdot b.
                        \end{equation}

              \end{enumerate}
        \item $\R$ jest zwarty, tj. każdy niepusty zbiór ograniczony z góry ma kres górny w $\R$.
    \end{enumerate}
\end{defr}

Można podać konstrukcję ciała o podanych własnościach (np. kontrukcja Dedekina, kontrukcja Riemanna) i dowieść, że z dokładnością do izomorfizmu istnieje tylko jedno takie ciało.

\begin{twier}{Własność Archimedesa}
    \begin{equation}
        \forall_{x > 0, y \in \R} \exists_{n \in \N}: nx > y.
    \end{equation}
\end{twier}


\begin{dow}
    Twiedzenia dowiedziemy nie wprost:\\
    Niech $X = {nx | n \in \N}$ i załóżmy, że X jest ograniczony z góry przez $y$. Zatem posiada supremum: $\alpha = \sup X$. Wiemy, że $\alpha - x < \alpha$, więc nie może to być ograniczenie górne. Zatem:
    \[ \exists_{n_0 \in \N}: n_0 x > \alpha - x. \]
    Ale wtedy:
    \[ X \ni (n_0 + 1) x > \alpha = \sup X. \]
    Sprzeczność! Istotnie więc, zbiór ${nx | n \in \N}$ nie może być ograniczony przez żadną liczbę, co dowodzi tezy.
\end{dow}

\begin{twier}{Gęstość $\Q$ w $\R$}
    \begin{equation}
        \forall_{x, y \in \R, \, x < y} \exists_{r \in \Q}: x < r < y.
    \end{equation}
\end{twier}
\begin{dow}
    Skoro $y - x > 0$, to $\exists{n \in \N}: n(y - x) > 1$. Ponadto, skoro $1 > 0$, to $\exists_{m_1, m_2 \in \N}: m_1 > n x \land m_2 > -nx$. Zatem $-m_2 < nx < m_1$, tzn. $nx$ leży pomiędzy dwiema liczami całkowitymi. Istnieje więc takie $m \in \Z$, takie że:
    \[ m-1 \leqslant nx < m. \]
    Stąd już prosto:
    \[nx < m \leqslant nx +1 < ny,\]
    \[x < \frac{m}{n} < y.\]
\end{dow}

\paragraph*{} Na koniec krótka notka - zbiór $\overline{\R} = \R \cup \{+\oo, -\oo\}$, to jest liczby rzeczywiste z dołączonymi symbolami (nie liczbami!) plus i minus nieskończoności, nazywami \textbf{rozszerzonymi liczbami rzeczywistymi}.


\newpage
\section{Ciagi rzeczywiste}
\subsection{Pojęcie ciągu i ogólne rezulataty}
\begin{defr}{Ciąg}
    \textbf{Ciagiem} elementów z zbioru $X$ nazywamy funkcję:
    \begin{equation}
        a: \N \to X
    \end{equation}
    i zamiast $a(n)$ piszemy $a_n$. Cały ciąg oznaczamy $(a_n)_{n \in \N}$. My w szczególności zajmować się będziemy ciągami rzeczywistymi i zespolonymi.
\end{defr}
W sekcji tej, o ile nie powiedziano inaczej, zakładamy, że wszystkie ciągi są rzeczywiste.
\begin{defr}{ZBIEŻNOŚĆ CIĄGU}
    Ciąg rzeczywisty $\ciag{a}$ nazywamy zbieżnym do granicy $g$, jeśli:
    \begin{equation}
        \forall_{\eps > 0} \exists_{N_{\eps} \in \N} \forall_{n \geqslant N}: |a_n - g| \leqslant \eps.
    \end{equation}
    Piszemy wtedy: $\lim_{n \to \oo} a_n = g$ lub $a_n \xrightarrow{n \to \oo} g$.
\end{defr}

Ciąg $\ciag{a}$ nazywamy \textbf{ograniczonym}, jeśli:
\begin{equation}
    \exists_{C} \forall_{n \in \N}: |a_n| \leqslant C.
\end{equation}

\begin{obs}{Obserwacja}
    Każdy ciąg zbieżny jest ograniczony.
\end{obs}.

\begin{twier}{Arytmetyka granic}
    Niech $\ciag{a}$ i $\ciag{b}$ będą ciągami rzeczywistymi, takimi, że $\lgras{n}{a} = a$ i $\lgras{n}{b} = b$. Wtedy:
    \begin{enumerate}
        \item \begin{equation}
                  \gras{n}{a + b} = a + b
              \end{equation}
        \item \begin{equation}
                  \gras{n}{a \cdot b} = a \cdot b
              \end{equation}
        \item \begin{equation}
                  \gras{n}{|a|} = |a|
              \end{equation}
        \item Jeśli $b_n \neq 0$ DDD $n$ (dla dostatecznie dużych $n$) i $b \neq 0$:
              \begin{equation}
                  \lim_{n \to \oo} \frac{a_n}{b_n} = \frac{a}{b}
              \end{equation}

    \end{enumerate}
\end{twier}

\begin{dow}
    W dowodach wszystkich tych twierdzeń chcemy dla dowolnego $\eps$ skonstruować takie $N$, że dla wszystkich $n \geqslant N$  różnica między wyrazami ciągu po lewej a granicą po prawej stronie jest mniejsza od $\eps$. Pamiętamy tutaj, że:
    \begin{equation}
        \forall_{\eps > 0} \exists_{M_{\eps} \in \N} \forall{n \geqslant M_{\eps}}: |a_n - a| \leqslant \eps,
    \end{equation}
    \begin{equation}
        \forall_{\eps > 0} \exists_{K_{\eps} \in \N} \forall{n \geqslant K_{\eps}}: |b_n - b| \leqslant \eps,
    \end{equation}

    \begin{enumerate}
        \item Mamy: \begin{equation}
                  |a_n + b_n - a - b| \leqslant |a_n - a| + |b_n - b|,
              \end{equation}
              więc dla $n \geqslant N = \max\{M_{\frac{\eps}{2}},\, K_{\frac{\eps}{2}}\}$: \begin{equation}
                  |a_n + b_n - a - b| \leqslant |a_n - a| + |b_n - b| \leqslant \frac{\eps}{2} + \frac{\eps}{2}  = \eps.
              \end{equation}

        \item \begin{equation}
                  | a_n b_n - ab| = |a_n b_n - a_n b + a_n b - ab| \leqslant |a_n| |b_n - b| + |b| |a_n -a|.
              \end{equation}
              Ale $\ciag{a}$ jest ograniczony, więc $a_n \leqslant C$ i mamy: \begin{equation}
                  | a_n b_n - ab|  \leqslant |a_n| |b_n - b| + b |a_n -a| \leqslant (|C| +1) |b_n - b| + (|b| +1) |a_n -a|.
              \end{equation}
              Zatem dla $n \geqslant \max\{M_{\frac{\eps}{2|b+1|}}, \, K_{\frac{\eps}{2(|C| +1 )}} \}$: \begin{equation}
                  | a_n b_n - ab| \leqslant (|C| +1) \frac{\eps}{2(|C| +1 )} + (|b| +1) \frac{\eps}{2|b+1|} = \eps.
              \end{equation}
              Dodaliśmy tutaj $1$ do $|C|$ i $|b|$, żeby uniknąć ewentualnego dzielenia przez $0$.


        \item Jeśli $a > 0$, to ciąg od pewnego miejsca musi być dodatni: $|a_n| = a_n$ dla $n > M_{|a|}$, więc dla $n > N = \max\{M_{|x|}, \, M_{\eps}\}$:
              \[ ||a_n| - |a|| = |a_n - a| < \eps. \]
              Podobnie, jeśli $a < 0$, to ciąg od pewnego miejsca jest ujemny i $|a_n| = -a_n$ dla $n > M_{|a|}$, więc dla $n > N = \max\{M_{|x|}, \, M_{\eps}\}$:
              \[ ||a_n| - a| = |-a_n - |a|| = |a_n -a| < \eps. \]
              Dla $a = 0$, mamy prosto:
              \[ |a_n| < \eps \implies ||a_n|| < \eps. \]

        \item Zakładamy, że $b_n \neq 0$ DDD $n$, więc istnieje takie $K_0$, że dla $n \geqslant K_0$ $b_n \neq 0$. Wtedy: \begin{equation}
                  \Big| \frac{a_n}{b_n} - \frac{a}{b} \Big| = \Big| \frac{a_n b - a b_n}{b_n b} \Big| = \Big| \frac{a_n b - ab + ab - a b_n}{b_n b} \Big| = \Big| \frac{b(a_n - a) + a(b -b_n)}{b_n b} \Big|
              \end{equation} \begin{equation}
                  \Big| \frac{a_n}{b_n} - \frac{a}{b} \Big| \leqslant \Big| \frac{a_n - a}{b_n} \Big| + \Big| \frac{a}{b_n b} \Big| \Big| b_n - b \Big| \leqslant \Big| \frac{a_n - a}{b_n} \Big| + (\Big| \frac{a}{b_n b} \Big| +1) \Big| b_n - b \Big|
              \end{equation}
              Zauważmy, że dla $n \geqslant K_{|\frac{b}{2}|}$ mamy $|b_n - b| \leqslant |\frac{b}{2}|$, więc $\frac{1}{2}|b| \leqslant |b_n|$, przez co: \begin{equation}
                  \Big| \frac{a_n}{b_n} - \frac{a}{b} \Big| \leqslant \Big| \frac{a_n - a}{b_n} \Big| + (\Big| \frac{a}{b_n b} \Big| +1) \Big| b_n - b \Big| \leqslant \Big| \frac{2}{b} \Big| \Big|a_n - a\Big| + (\Big| \frac{2 a}{b} \Big| +1) \Big| b_n - b \Big|
              \end{equation}

              Ostatecznie dla $n > \max\{K_0, K_{|\frac{b}{2}|}, K_{\frac{\eps}{2 (| \frac{2 a}{b} | +1)}}, M_{\frac{\eps}{4|b|}} \}:$ \begin{equation}
                  \Big| \frac{a_n}{b_n} - \frac{a}{b} \Big| \leqslant \Big| \frac{2}{b} \Big| \frac{\eps}{4|b|} + (\Big| \frac{2 a}{b} \Big| +1) \frac{\eps}{2 (| \frac{2 a}{b} | +1)}  = \eps.
              \end{equation}


    \end{enumerate}
\end{dow}

\paragraph{} Powiemy teraz o mocnym twierdzeniu, pozwalającym stwierdzić, czy ciąg ma granicę, bez jej wyznaczania.
\begin{twier}{Twierdzenie o ciągu monotonicznym i ograniczonym}
    Każdzy ciąg monotoniczny i ograniczony jest zbieżny. \begin{itemize}
        \item Jeśli ciąg jest niemalejący, to jest on zbieżny do supremum zbioru wyrazów ciągu.
        \item Jeśli ciąg jest nierosnący, to jest on zbieżny do infimum zbioru wyrazów ciągu.

    \end{itemize}
    Uwaga - dla ciągu $\ciag{a}$ supremum jego wyrazów - tj. $\sup \{ a_n \, | \, n \in \N \}$ - oznaczamy $\mathlarger{\sup}_{n \in \N} \, a_n$. Analogicznie piszemy $\mathlarger{\inf}_{n \in \N} \, a_n$ dla infimum jego wyrazów.
\end{twier}

\begin{dow}
    Załóżmy, że $\ciag{a}$ jest niemalejący. Zbiór $\{ a_n \, | \, n \in \N \}$ jest ograniczony, zatem posiada supremum. Oznaczmy je $g$. Zatem dla każdego $\eps > 0$ liczba $g - \eps$ nie jest ograniczeniem górnym: \begin{equation}
        \exists_m: g - \eps \leqslant a_m \leqslant g.
    \end{equation}
    Ale wtedy, z racji monotoniczności $\ciag{a}$: \begin{equation}
        \forall_{n \geqslant m}: g - \eps \leqslant a_m \leqslant a_n \leqslant g \leqslant g + \eps.
    \end{equation}
    Czyli:  \begin{equation}
        \forall_{n \geqslant m}: |a_n - g| \leqslant \eps,
    \end{equation}
    co chcieliśmy pokazać. Dla $\ciag{a}$ nierosnącego dowód jest zupełnie analogiczny (można też rozważać zbieżność niemalejącego ciągu $\ciag{-a}$).

\end{dow}


\begin{twier}{Twiedzenie  o trzech ciągach}
    Niech $\ciag{a}$ i $\ciag{c}$ będą dwoma ciągami zbieżnymi do wspólnej granicy $g$. Wtedy, jeśli dla ciągu $\ciag{b}$ istnieje takie $N$, że: \begin{equation}
        \forall_{n \geqslant N}: a_n \leqslant b_n \leqslant c_n,
    \end{equation}
    to $\lgras{n}{b} = g$.
\end{twier}

\begin{dow}
    Dowód jest bardzo krótki. Dla dowolnego $\eps$ bierzemy takie $M_{\eps}$, że $\forall_{n \geqslant M_{\eps}}: |a_n - g| \leqslant \eps $ i takie $K_{\eps}$, że: $\forall_{n \geqslant K_{\eps}}: |c_n - g| \leqslant \eps $. Wtedy dla $n \geqslant \max\{N, M_\eps, K_\eps \}$: \begin{equation}
        g - \eps \leqslant a_n \leqslant b_n \leqslant c_n \leqslant g - \eps,
    \end{equation}
    więc $|b_n - g| \leqslant \eps$.
\end{dow}



\begin{defr}{Rozbieżność do $\pm \oo$}
    Powiemy, że ciąg $\ciag{a}$ jest rozbieżny do $+\oo$, jeśli: \begin{equation}
        \forall_C \exists_{N_C \in \N} \forall_{n \geqslant N_C}: a_n \geqslant C.
    \end{equation}
    Analogicznie, powiemy, że ciąg $\ciag{a}$ jest rozbieżny do $-\oo$, jeśli: \begin{equation}
        \forall_C \exists_{N_C \in \N} \forall_{n \geqslant N_C}: a_n \leqslant C.
    \end{equation}
    Mówimy też o "zbieżności" do $\pm \oo$, tj. zbieżności w zbiorze $\overline{\R}$
\end{defr}

\begin{obs}{}
    Ciąg monotoniczny, nieograniczony jest rozbieżny do $\pm \oo$.
\end{obs}

\begin{defr}{Podciąg}
    Podciągiem ciagu $\ciag{a}$ nazywamy ciąg $(a_{n_k})_{k \in \N}$, gdzie $n \ni k \mapsto n_k \in \N$ jest funkcją ściśle rosnącą.
\end{defr}


\begin{obs}{}
    Jeśli $a_n \to g$, to każdy podciąg $a_{n_k} \to g$
\end{obs}





\subsection{$\limsup$ i $\liminf$}
%%%%10 X 2023%%%



\paragraph{}
Załóżmy, że mamy dany ciąg liczb rzeczywistych $\ciag{a}$. Zdefiniujmy wtedy następujące dwa ciągi: $\ciag{\alpha}$ oraz $\ciag{\beta}$, takie że: \begin{equation}
    \alpha_n = \inf \{ a_k\, |\, k \geqslant n \}
\end{equation} \begin{equation}
    \beta_n = \sup \{ a_k \, | \, k \geqslant n \}
\end{equation}
Wtedy, $\ciag{\alpha}$ jest ciągiem niemalejącym, co wynika z faktu, że:
\[ \inf \{ a_k\, |\, k \geqslant n + 1 \} \, \subseteq \, \inf \{ a_k\, |\, k \geqslant n \} \]
Podobnież, $\ciag{\beta}$ jest ciągiem nierosnącym. Z tego wynika więc, że są to ciągi zbieżne w $\overline{\R}$.
Mamy więc: \begin{equation}
    \lim_{n \to \infty} \alpha_n = \sup \alpha_n,
\end{equation}
gdyż $\ciag{\alpha}$ to ciąg nierosnący oraz podobnie: \begin{equation}
    \lim_{n \to \infty} \beta = \inf \beta.
\end{equation}

\begin{defr}{Granice górne i dolne ciągu}
    Wielkość: \begin{equation}
        \lim_{n \to \oo} \inf \{ a_k\, |\, k \geqslant n \} = \gras{n}{\alpha} = \sup{\alpha_n}
    \end{equation} nazywamy $\textbf{granicą dolną}$ ciągu $\ciag{a}$ i oznaczamy $\mathlarger{\liminf}_{n \to \oo} a_n$. Podobnie, wielkość: \begin{equation}
        \lim_{n \to \oo} \sup \{ a_k\, |\, k \geqslant n \} = \gras{n}{\beta} = \inf{\beta_n}
    \end{equation} nazywamy $\textbf{granicą górną}$ ciągu i oznaczamy $\mathlarger{\limsup}_{n \to \oo} a_n$.
\end{defr}


\begin{twier}{Bolzano-Weierestrassa I} \label{twier:BL I}
    Niech $L$ będzie zbiorem punktów skupienia zbioru $\{a_n \, | \, n \in \N\}$, tzn. takich liczb, dla których istnieje podciąg ciągu $(\ciag{a}$ zbieżny do tej liczby: \begin{equation}
        L = \{ x \in \overline{\R} \,\, | \,\, \exists_{\text{podciąg} \, (a_{n_k})_{k \in \N}} : \, \lim_{k \to \infty} a_{n_k} = x \}.
    \end{equation}
    Wtedy:
    \begin{enumerate}
        \item \begin{equation}
                  L \neq \emptyset
              \end{equation}
        \item \begin{equation}
                  \liminf{a_n} \in L  \quad \land \quad \limsup{a_n} \in L
              \end{equation}
        \item \begin{equation}
                  \liminf{a_n} = \inf{L} \quad \land \quad \limsup{a_n} = \sup{L}
              \end{equation}
    \end{enumerate}
\end{twier}

\begin{dow}
    TODO
\end{dow}

Nietrudnym wnioskiem z tego twierdzenia jest następujące:
\begin{twier}{Kryterium zbieżności}
    Ciąg $\ciag{a}$ jest zbieżny do $a \in \Rbar$ wtedy i tylko wtedy, gdy: \begin{equation}
        \liminf a_n = \limsup a_n.
    \end{equation}
\end{twier}

\begin{dow}
    \paragraph*{$\implies$} Jeśli $\gras{n}{a} = a$, to także każdy podciąg $\ciag{a}$ dąży do $a$, więc $L = \{ a \}$. Ale $\liminf a_n \in L$ i $\limsup a_n \in L$, więc: \begin{equation}
        \liminf a_n = a = \limsup a_n.
    \end{equation}
    \paragraph*{$\impliedby$} Oczywiście zachodzi nierówność: \begin{equation}
        \alpha_n \leqslant a_n \leqslant \beta_n.
    \end{equation}
    Skoro mamy: \begin{equation}
        \gras{n}{\alpha} = \liminf a_n = \limsup a_n = \gras{n}{\beta},
    \end{equation}
    więc z twierdzenia o trzech ciągach mamy: \begin{equation}
        \gras{n}{a} = \liminf a_n = \limsup a_n.
    \end{equation}
\end{dow}

\begin{twier}{Warunek Cauchy'ego}
    Niech $\ciag{a}$ będzie ciągiem rzeczywistym. Wtedy następujące warunki są rónoważne:
    \begin{enumerate}
        \item \begin{equation}
                  \gras{n}{a} = a \in \R
              \end{equation}
        \item \begin{equation}
                  \forall_{\eps >0} \exists_{M_\eps} \forall_{n, m \geqslant M_\eps} |a_n - a_m| \leqslant \eps.
              \end{equation}
              Jeśli ciąg spełnia ten warunek, mówimy, że spełnia warunek Cauchy'ego.
    \end{enumerate}
\end{twier}

Sprawdzając warunek Cauchy'ego, możemy dowodzić zbieżności ciągu do granicy rzeczywistej bez wyznaczania tej granicy.

\begin{dow}
    \paragraph*{1. $\implies$ 2.} Skoro $\ciag{a}$ zbieżny do $a \in R$, to dla dowolnego $\eps >0$ istnieje takie $N_{\frac{\eps}{2}}$, że: \begin{equation}
        \forall_{n, m \geqslant N_{\frac{\eps}{2}}}: \,|a_n - a| \leqslant \frac{\eps}{2} \,\, \land\,\, |a_m - a| \leqslant \frac{\eps}{2},
    \end{equation}
    ale wtedy: \begin{equation}
        |a_n - a_m| \leqslant \frac{\eps}{2} + \frac{\eps}{2} = \eps.
    \end{equation}

    \paragraph*{2. $\implies$ 1.} Zauważmy, że $\ciag{a}$ jest ciągiem ograniczonym. Istotnie, weźmy $\eps = 1$: \begin{equation}
        \exists_{M_1} \forall_{n \geqslant M_1}: |a_n - a_{M_1}| \leqslant 1,
    \end{equation} więc: \begin{equation}
        \min \{ a_{M_1} -1; a_k \, \big| \, k < M_1 \} \leqslant a_n \leqslant \max \{ a_{M_1} +1; a_k \, \big| \, k < M_1 \}.
    \end{equation}
    Weźmy teraz dowolny $\eps > 0$. Z ograniczoności $\ciag{a}$ wynika: $\alpha = \liminf a_n \in \R$ i $\beta = \limsup a_n \in \R$. Oznacza to, że: \begin{equation}
        \forall_{\eps > 0} \exists_{A_\eps} \forall_{n \geqslant A_\eps}: |\alpha_n - \alpha| \leqslant \eps,
    \end{equation} \begin{equation}
        \forall_{\eps > 0} \exists_{B_\eps} \forall_{n \geqslant B_\eps}: |\beta_n - \beta| \leqslant \eps,
    \end{equation}
    Zauważmy, że: \begin{equation}
        |\alpha - \beta| \leqslant |\alpha - \alpha_n| + |\beta - \beta_n| + |\alpha_n - \beta_N|.
    \end{equation}
    Dwa pierwsze czynniki po prawej potrafimy ograniczyć, zbadajmy więc ostatni wyraz: \begin{equation} \begin{split}
            |\alpha_n - \beta_N| \leqslant |\alpha_n - a_n| + |\beta_n - a_n| & = |\inf\{a_n, a_{n+1}, ...\} - a_n| + |\sup\{a_n, a_{n+1}, ...\} - a_n| \\
            & = \inf\{ |a_m - a_n| \, \big| \, m \geqslant n \} + \sup\{ |a_m - a_n| \, \big| \, m \geqslant n \}.
        \end{split} \end{equation}
    Możemy teraz skorzystać z warunku Cauchy'ego i znaleźć takie $M_{\frac{\eps}{6}}$, że: \begin{equation}
        \forall_{m \geqslant n \geqslant M_{\frac{\eps}{6}}}: |a_m - a_n| \leqslant \frac{\eps}{6}.
    \end{equation} Zatem dla $n \geqslant M_{\frac{\eps}{6}}$:\begin{equation}
        \inf\{ |a_m - a_n| \, \big| \, m \geqslant n \} \leqslant \frac{\eps}{6}
    \end{equation} \begin{equation}
        \sup\{ |a_m - a_n| \, \big| \, m \geqslant n \} \leqslant \frac{\eps}{6}
    \end{equation}
    Ostecznie otrzymujemy dla $n \geqslant \max \{A_{\frac{\eps}{3}}, \, B_{\frac{\eps}{3}}, \, M_{\frac{\eps}{6}}\}$: \begin{equation}
        |\alpha - \beta| \leqslant |\alpha - \alpha_n| + |\beta - \beta_n| +  |\alpha_n - a_n| + |\beta_n - a_n| \leqslant \frac{\eps}{3} + \frac{\eps}{3} + \frac{\eps}{6} + \frac{\eps}{6} = \eps.
    \end{equation}
    Pokazaliśmy, że różnica $|\liminf a_n - \limsup a_n|$ jest mniejsza od dowolnej liczby dodatniej, zatem musi być równa $0$. Oznacza to, że $\liminf a_n = \limsup a_n \in \R$, a więc - co udowodniliśmy już wcześniej - ciąg jest zbieżny do rzeczywiśtej granicy.
\end{dow}



\newpage
\section{Szeregi liczbowe}
\begin{defr}{Szereg liczbowy}
    Niech $\ciag{a}$ będzie ciągiem liczb zespolonych ($a_n \in \C$). \textbf{Szeregiem} o wyrazach $\ciag{a}$ nazwiemy napis: \begin{equation}
        \bsum{n=1}{\oo} a_n.
    \end{equation} \textbf{$N$-tą sumą częściową} szeregu $\szerI{a}$ nazwiemy sumę: \begin{equation}
        S_N = \bsum{n=1}{N} a_n.
    \end{equation} Powiemy, że szereg $\szerI{a}$ jest \textbf{zbieżny} do $S$, jeśli: \begin{equation}
        \gras{N}{S} = S \in \C.
    \end{equation} Napiszemy wtedy, że $\szerI{a} = S$, a liczbę $S$ nazwiemy sumą szeregu. Jeśli $S_N$ nie ma granicy, to szereg nazwiemy \textbf{rozbieżnym}.
\end{defr}

\paragraph*{Przykład 1.} Jeśli $a_n = (-1)^n$, to szereg $\szerI{a}$ jest rozbieżny.
\paragraph*{Przykład 2.} Jeśli $a_n = q^{n-1}$, dla $|q| < 1$, to $S_N = \frac{1 - q^n}{1 - q}$ i $\szerI{q^{n-1}} = \frac{1}{1-q}$.


\begin{twier}{} \label{twier:ogon-szer}
    Szereg $\szerI{a}$ jest zbieżny wtedy i tylko wtedy, gdy: \begin{equation}
        \forall_\eps \exists_{N_\eps} \forall_{N \geqslant M \geqslant N_\eps} | \bsum{n= M+1}{N} a_n | \leqslant \eps.
    \end{equation} 
\end{twier}
\begin{dow}
    Twierdzenie wynika prosto z zastosowania warunku Cauchy'ego do ciągu $S_N$. Bowiem $S_N$ jest zbieżny wtedy i tylko wtedy, gdy:\begin{equation}
        \forall_\eps \exists_{N_\eps} \forall_{N \geqslant M \geqslant N_\eps} |S_N - S_m| \leqslant \eps,
    \end{equation} co po rozpisaniu $S_N$ i $S_M$ jest równoważne twierdzeniu.
\end{dow}
\paragraph*{} $\bsum{n= M+1}{N} a_n$ nazywa się czasami \textbf{ogonem szeregu}. 
\begin{obs}{Warunek konieczny zbieżności}
    Warunkiem koniecznym zbieżności szeregu $\szerI{a}$ jest, aby $\gras{n}{a} = 0$.
\end{obs}
\paragraph{} Istotnie, wystarczy podstawić $M = N-1$ w twierdzeniu \ref{twier:ogon-szer}, żeby otrzymać definicję zbieżności $a_n$ do 0.
\paragraph{} Warto pamiętać, że nie jest to warunek wystarczający - kontrprzykładem jest np. rozbieżny szereg $\szerI{\frac{1}{n}}$. Wtedy\begin{equation}
    S_{2^k} = 1 + \bsum{l = 1}{k} \bsum{j = 2^{l-1} +1}{2^{l}} \frac{1}{j} \geqslant1 +  \bsum{l = 1}{k} \frac{2^{l-1}}{2^l} = 1 + \half k \to \oo. 
\end{equation} Zatem i $S_N \to \oo$, bo jest to ciąg monotoniczny.

\subsection{Szeregi o wyrazach dodatnich}
%%%17 X 2023%%%

\paragraph{} W tej sekcji zajmiemy się jedynie szeregami o wyrazach dodatnich, to jest takimi, dla których $\R \ni a_n \geqslant 0$. 
\begin{obs}{} \label{obs:szer-dod}
    Dla szeregów o wyrazach dodatnich, ciąg sum częściowych jest niemalejącym ciągiem rzeczywistym. Oznacza to, że szereg jest zbieżny wtedy i tylko wtedy, gdy ciąg sum częściowych jest ograniczony. W przeciwnym wypadku $\gras{N}{S} = + \oo$ i wtedy mówimy, że $\szerI{a}$ jest rozbieżny do $+\oo$. 
\end{obs}

\begin{twier}{Kryterium porównawcze} \label{twier:kryt-por}
    Niech $\szerI{a}$ ,$\szerI{b}$ będą szeregami o wyrazach dodatnich. Wtedy: \begin{enumerate}
        \item Jeśli $\exists_{C > 0}: a_n \leqslant C b_n$ DDD$n$, to: \begin{equation}
            \szerI{b} < \oo \implies \szerI{a} < \oo.
        \end{equation}
        \item Jeśli $\exists_{C > 0}: a_n \geqslant C b_n$ DDD$n$, to: \begin{equation}
            \szerI{b} = +\oo \implies \szerI{a} = +\oo.
        \end{equation}
    \end{enumerate}
\end{twier}

\begin{dow}
    Niech $N_0$ będzie oznaczał indeks, od którego podane nierówności zachodzą.
    \begin{enumerate}
        \item Dla $N > N_0$ mamy: \begin{equation*}
            S_N^{(a)} = S_{N_0}^{(a)} + \bsum{n = N_0 +1}{N}a_n \leqslant S_{N_0}^{(a)} + \bsum{n = N_0 +1}{N}C b_n = S_{N_0}^{(a)} - C S_{N_0}^{(b)} + C S_N^{(b)} \leqslant S_{N_0}^{(a)} - C S_{N_0}^{(b)} + C \szerI{b} < \oo. 
        \end{equation*} Zatem skoro $S_N^{(a)}$ jest ograniczony, to (por. Obs. \ref{obs:szer-dod}) jest i zbieżny.
        \item Podobnie jak poprzednio, dla $N > N_0$ mamy: \begin{equation*}
            S_N^{(a)} = S_{N_0}^{(a)} + \bsum{n = N_0 +1}{N}a_n \geqslant S_{N_0}^{(a)} + \bsum{n = N_0 +1}{N}C b_n = S_{N_0}^{(a)} - C S_{N_0}^{(b)} + C S_N^{(b)} \xrightarrow[]{N \to \oo} \oo,
        \end{equation*} więc i $S_N^{(a)}$ jest rozbieżny do nieskończoności.
    \end{enumerate}
\end{dow}


\begin{twier}{Kryterium porównawcze (wersja graniczna)}
    Niech $\szerI{a}$ ,$\szerI{b}$ będą szeregami o wyrazach dodatnich i niech $\mathlarger{\lim}_{n \to \oo} \frac{a_n}{b_n} = L$. Wtedy: \begin{enumerate}
        \item Jeśli $L < \oo$, to: \begin{equation}
            \szerI{b} < \oo \implies \szerI{a} < \oo.
        \end{equation}
        \item Jeśli $ L > 0$, to: \begin{equation}
            \szerI{b} = +\oo \implies \szerI{a} = +\oo.
        \end{equation}
    \end{enumerate}
\end{twier}


\begin{dow}
    \begin{enumerate}
        \item Jeśli $L < \oo$, to $a_n \leqslant  (L + L) * b_n$ dla dostatecznie dużyn $n$ i stosujemy  Tw. \ref{twier:kryt-por}.1.
        \item Jeśli $L > 0$, to $a_n \leqslant \half L b_n$ DDD$n$ (ew. $a_n \leqslant 2 b_n$, jeśli $L = \oo$) i stosujemy Tw. \ref{twier:kryt-por}.2.
    \end{enumerate}
\end{dow}


\begin{twier}{Kryterium D'Alemberta}
    Niech $\forall_n a_n > 0$ i niech $d_n = \frac{a_{n+1}}{a_n}$. Wtedy: \begin{enumerate}
        \item \begin{equation}
            \limsup d_n < 1 \implies \szerI{a} < \oo
        \end{equation}
        \item \begin{equation}
            (\liminf d_n > 1 \vee \exists_k \forall_{n \geqslant k}: d_n \geqslant 1) \implies \szerI{a} = \oo.
        \end{equation} Ponadto, jeśli zachodzi poprzecznik implikacji, to $a_n \not \to 0$.
    \end{enumerate}
\end{twier} 
\begin{dow}
    \begin{enumerate}
        \item Skoro $\limsup d_n < 1$, to $d_n$ musi być ograniczony. Niech $\lambda$ będzie taką liczbą, że $1 > \lambda > \limsup d_n$. Zauważmy, że może istnieć tylko skończona liczba wyrazów $d_n$ większych lub równych $\lambda$, gdyż inaczej wybralibyśmy z nich podciąg zbieżny do granicy $\geq \lambda$, co przeczy założeniu, że $\lambda < \limsup d_n$. Zatem: \begin{equation*}
            \exists_{N} \forall_{n \geq N}: d_n < \lambda.
        \end{equation*} Czyli: \begin{equation*}
            \exists_{N} \forall_{n \geq N}: a_{n+1} < \lambda a_n.
        \end{equation*} Stąd, dla $n > N$: $a_{n} < \frac{a_N}{\lambda^N} \lambda^n$. Tak więc na mocy kryterium porównawczego ze zbieżnym szeregiem $\bsum{n=1}{\oo}\lambda^n$ ($\lambda < 1$) szereg $\szerI{a}$ jest zbieżny.

        \item Jeśli $\liminf d_n > 1$, to (używając podobnego rozumowania, jak poprzednio) istnieje $\rho > 1$ t.ż. $d_n > \rho$. Skoro tak, to DDD$n$: $a_{n+1} > \rho a_n > a_n$ i $a_n$, jako rosnący ciąg o wyrazach dodatnich, nie może dążyć do 0. Podobnie, jeśli DDD$n$ mamy $d_n \geq 1$, to $a_{n+1} \geq a_n$ i znowuż $a_n$ nie może dążyć do $0$.
    \end{enumerate}
\end{dow}

\paragraph{Uwagi.} \begin{enumerate}
    \item Może być tak, że $d_n \geq 1$ dla nieskończenie wielu $n$, a szereg $\szerI{a}$ jest zbieżny. Przykład: \begin{equation*}
        a_n = \begin{cases}
            \frac{1}{(\frac{n}{2})^2}, & \text{gdy } n  = 2k \\
            \frac{1}{(\frac{n-1}{2})^2}, &\text{gdy } n = 2k+ 1
        \end{cases}
    \end{equation*}
    \item Może być tak, że szereg $\szerI{a}$ jest zbieżny, ale $\limsup d_n > 1$. 
\end{enumerate}


Następne kryterium jest rozszerzeniem kryterium D'Alemberta: 
%TODO
\begin{twier}{Kryterium Cauchy'ego}
TODO
    
\end{twier}


\begin{twier}{}
    Załóżmy, że $\sum_{n=1}^{\infty} a_n$ jest szeregiem o wyrazach dodatnich ($a_n > 0$). Wtedy:
    \begin{equation}
        \sum\limits_{n=1}^{\infty} a_n = \sup_{\substack{F \subset \N \\ |F| < +\infty}} \sum_{n \in F} a_n.
    \end{equation}
\end{twier}

\begin{dow}
    Oczywiście, skoro $S_N$ jest rosnący, to $\szerI{a} = \gras{N}{S} = \sup_{N \in \N} S_N$. Łatwo widać też, że $\sup_{N \in \N} S_N \leq \sup_{\substack{F \subset \N \\ |F| < +\infty}} \sum_{n \in F} a_n$ (ponieważ $S_N$ to suma tego samego typu, co po prawej stronie z $F = \{1, ..., N\}$). Dodatkowo, dla każdego skończonego $F \se \N$, istnieje takie $N$, że $\forall_{n \in F} x \leq N$, zatem $\sum_{n \in F} a_n \leq S_N$. Ostatecznie mamy $ \sup_{\substack{F \subset \N \\ |F| < +\infty}} \sum_{n \in F} a_n \leq \sup_{N \in \N} S_N$, stąd te dwie wielkości są sobie równe i mamy tezę.
\end{dow}



\paragraph{} Wynikają z tego następujące wnioski:
\begin{twier}{}
    \begin{enumerate}
        \item \[\sum_{n = 1}^{\infty} a_n < + \infty \quad \iff \quad \exists_{C>0} \forall_{F \subset \N ,\, |F| < +\oo}  \sum_{n \in F} a_n \leqslant C\]
              (Zbiór wszystkich sum elementów o indeksach pochodzących ze SKOŃCZONEGO podzbioru $\N$ jest ograniczony)
        \item Jeśli $\sigma : \N \rightarrow \N$ jest bijekcją (permutacją indeksów), to:
              \[\sum_{n=1}^{\oo} a_{\sigma(n)} = \sum_{n=1}^{\oo} a_{n}.\]
              Oznacza to, że zmiana kolejności sumowania nie wpływa na wynik.
        \item Jeśli:
              \[\N = \bigsqcup_{\substack{i=1}}^{\oo} A_i \quad (\text{jest to suma rozłączna, tj. } A_i \cap A_j = \emptyset \text{ dla } i \neq j)\]
              i:
              \[S_i = \sum_{n \in A_i} a_n,\]
              to:
              \[\sum_{i=0}^{\oo}S_i  = \sum_{n=1}^{\oo}a_n.  \]
              Jest to grupowania (łączności i przemienności) dla szeregu.
    \end{enumerate}
\end{twier}
\begin{dow}
    \begin{enumerate}
        \item Ograniczoność sum po prawej jest oczywiście równoważne istnieniu skończonego ich supremum, co równe jest sumie po lewej.
        \item $\sigma$ zachowuje klasę skończonych podzbiorów i wyznacza bijekcję:
              \[\sigma: 2^{\N} \rightarrow 2^{\N}\]
              \[\sigma(F) = \{ \sigma(n)\,|\, n \in F \},\]
              która zachowuje moc zbioru $F$ i przeprowadza zbiory skończone na skończone. Zatem:
              \[\biggl\{ \sum_{n \in F} a_n \, \bigg| \, F \subset \N, \, |F| < \oo\biggr\} = \biggl\{ \sum_{n \in \sigma(F)} a_{n} \, \bigg| \, F \subset \N, \, |F| < \oo\biggr\}, \]
              więc:
              \[ \sum_{n=1}^{\oo} a_n = \sup{\biggl\{ \sum_{n \in F} a_n \, \bigg| \, F \in 2^{\N}\biggr\}}  = \sup{\biggl\{ \sum_{n \in \sigma(F)} a_{n} \, \bigg| \, F \subset \N, \, |F| < \oo\biggr\}} = \sum_{n=1}^{\oo} a_{\sigma(n)}\]
        \item
              \textbf{a)} Jeśli $\exists_j: \, S_j = \oo$, to:
              \[\sum_{i=1}^{\oo} S_i = \oo\]
              i:
              \[\sum_{n=1}^{\oo} a_n \geqslant \sum_{n \in A_j} a_n = \oo.\]
              Stąd:
              \[\sum_{i=1}^{\oo} S_i = \oo = \sum_{n=1}^{\oo} a_n\]
              \textbf{b)} Niech $\forall_{i}: \, S_i < \oo$. Ustalmy $\varepsilon > 0$ i weźmy dowolny:
              \[K \subset \N, \quad |K| < \oo.\]
              Przypomnijmy:
              \[\sum_{i=1}^{\oo}S_i = \sup_{\substack{K \subset \N, \\ |K| < \oo}} \, \sum_{j \in K} S_j.\]
              Niech $K = \{ i_1, i_2,\, ...\,, i_l \}$ i wybierzmy:
              \[C_1 \subseteq A_{i_1}, \, C_2 \subseteq A_{i_2}, \, ...\, , \, C_l \subseteq A_{i_l}, \]
              takie, że:
              \[\forall_j: \quad \sum_{n \in C_j} a_n \,  \geqslant \, S_{i_j} - \frac{\varepsilon}{l},\]
              co jest możliwe, gdyż $S_{i_j}$ jest supremum sum po skończonych podzbiorach. Jeśli $A_{i_j}$ jest skończony, możemy przyjąć $C_j = A_{i_j}$. \\
              Wtedy:
              \[\sum_{i \in K} S_i = S_{i_1} + S_{i_2} + \, ... \, + S_{i_l} \leqslant \left( \sum_{n\in C_1} a_n + \frac{\varepsilon}{l} \right) + \left( \sum_{n\in C_l} a_n + \frac{\varepsilon}{l} \right) + \, ... \, \left( \sum_{n\in C_l} a_n + \frac{\varepsilon}{l} \right) = \varepsilon + \sum_{n \in \bigcup_{j=1}^l C_j} a_n.\]
              NB: $\bigcup_{j=1}^l C_j$ jest zbiorem skończonym, więc:
              \[\sum_{i \in K} S_i \leqslant \varepsilon + \sup_{\substack{F\subset \N \\ |F| <  +\oo}} \sum_{n \in F} a_n = \varepsilon + \sum_{n=1}^{\oo} a_n.\]
              Rozumowanie to przeprowadziliśmy dla dowolnego $K$ i $\varepsilon$, więc:
              \[\sum_{i=1}^{\oo} S_i = \sup_{\substack{K \subset \N \\ |K| < \oo}} \sum_{i \in K} S_i \leqslant \sum_{n=1}^{\oo} a_n.\]
              W drugą stronę, weźmy $F \subset \N$, $|F| < \oo$ i niech:
              \[K = \{ i \in \N \, |\, A_i \cap F \neq \emptyset \},\]
              wtedy $|K| < \oo$, bo $F$ jest skończony, a $A_j$ są rozłączne. Wtedy mamy:
              \[\sum_{n\in F} a_n = \sum_{i\in K} \sum_{n \in A_i \cap F} a_n \leqslant \sum_{i \in K} \sum_{n \in A_i} a_n = \sum_{i \in K} S_i.\]
              Z tego:
              \[\sum_{n \in F} a_n \leqslant \sum_{i \in K} S_i \leqslant \sup_{\substack{K \subset \N \\ |K| < \oo}} \sum_{i \in K} S_i = \sum_{i =1}^{\oo} S_i, \]
              co zachodzi dla dowolnego $F$ skończonego. Zatem:
              \[\sup_{\substack{F \subset \N \\ |F| < \oo}}\sum_{n \in F} a_n = \sum_{n=1}^{\oo}a_n \leqslant \sum_{i =1}^{\oo} S_i.\]
              Porównując dwie otrzymane nierówności, otrzymujemy tezę.
    \end{enumerate}
\end{dow}

\subsection{Szeregi o wyrazach dowolnych}
W tej podsekcji rozważamy szeregi o dowolnych wyrazach zespolonych:
\[\sum_{n=1}^{\oo} z_n, \quad z_n \in \C.\]
\begin{twier}{Kryterium zbieżności bezwzględnej}
    Jeśli następujący szereg jest zbieżny:
    \[\sum_{n=1}^{\oo} |z_n| < + \oo\] to i:
    \[\sum_{n=1}^{\oo} z_n\]
    jest zbieżny.

\end{twier}
\begin{dow}
    Korzystamy z warunku Cauchy'ego dla szeregu modułów:
    \[\forall_{\varepsilon}\exists_{N_{\varepsilon}}\forall_{n > m\geqslant N_{\varepsilon}}:\, |\sum_{k=m+1}^{n} |z_k|| \leqslant \varepsilon,\]
    ale:
    \[|\sum_{k=m+1}^{n} z_k| \leqslant |\sum_{k=m+1}^{n} |z_k||,\]
    więc:
    \[\forall_{\varepsilon}\exists_{M_{\varepsilon} = N_{\varepsilon}}\forall_{n > m\geqslant M_{\varepsilon}}:\, |\sum_{k=m+1}^{n} z_k| \leqslant \varepsilon.\]
    To dowodzi, że warunek Cauchy'ego zachodzi dla $\sum_{n=1}^{\oo}z_n$, zatem jest to szereg zbieżny.
\end{dow}
\begin{defr}{Zbieżność bezwzględna}
    \begin{itemize}
        \item Szereg $\mathlarger{\sum}_{n=1}^{\oo} z_n$ nazywamy zbieżnym bezwzględnie, jeśli szereg $\mathlarger{\sum}_{n=1}^{\oo} |z_n|$ jest zbieżny. \\
        \item Jeśli szereg $\mathlarger{\sum}_{n=1}^{\oo} z_n$ jest zbieżny, ale $\mathlarger{\sum}_{n=1}^{\oo} |z_n|$ już nie, to szereg nazywamy zbieżnym warunkowo.
    \end{itemize}
\end{defr}

Dla szeregów o wyrazach dowolnych obowiązują inne kryteria zbieżności niż dla szeregów o wyrazach dodatnich. Jednym z nich, jest:
\begin{twier}{Kryterium Dirichleta}
    Załóżmy, że:
    \begin{itemize}
        \item Mamy ciągi:
              \[(a_n)_{n\in \N}, \quad a_n \in \C \]
              \[(b_n)_{n\in \N}, \quad b_n \in \R, \quad b_n \geqslant 0\]
        \item $b_n$ zbiega monotonicznie do 0.
        \item Ciąg sum częściowych wyrazów $(a_n)$ jest ograniczony:
              \[\exists_{C \in \R} \forall_{N \in \N}: \big| \sum_{n=1}^{N} a_n \big| \leqslant C.\]
    \end{itemize}
    Wtedy szereg $\mathlarger{\sum}_{n=1}^{\oo} a_n b_n$ jest zbieżny.
\end{twier}
\paragraph*{Uwaga!} W kryterium tym wystarczy, żeby $b_n$ był nierosnący (i.e. nie trzeba, by był on ściśle malejący).
\paragraph{Przykład - szeregi naprzemienne.} Weźmy szereg postaci:
\[\sum_{n=1}^{\oo} (-1)^n b_n,\]
gdzie $b_n \searrow 0$ (dąży monotonicznie z góry do 0). Szereg taki nazywamy szeregiem naprzemiennym. Z kryterium Dirichleta wynika, że każdy szereg takiej postaci jest zbieżny (co nazywa się czasem kryterium Leibnitza). W szczególności:
\[\sum_{n=1}^{\oo} \frac{(-1)^n}{n} = \ln 2.\]

\begin{dow}[Dowód kryterium Dirichleta.]
    Niech $\sum_{n=1}^{N} a_n = z_N$. Zauważmy, że $(z_N)$ jest ciągiem ograniczonym. Zapiszmy sumy częściowe docelowego szerego:
    \[\sum_{n=1}^{N} a_n b_n = a_1 b_1 + a_2 b_2 + \, ... \, + a_N b_N = z_1 b_1 + (z_2 - z_1) b_2 + \, ... \, +  (z_N - z_{N-1}) b_N = \]
    \[ = z_1 (b_1 - b_2) + z_2 (b_2 - b_3) + \, ... \, + z_{N-1} (b_{N-1} - b_N) + z_N b_N.\]
    Zauważmy, że wyraz $z_N b_N$ jest zbieżny do 0 (jako iloczyn czynnika ograniczonego i czynnika dążącego do 0).
    Zajmijmy się więc otrzymaną sumą. Zauważmy, że zachodzi:
    \[\sum_{n=1}^{\oo} | z_n (b_n - b_{n+1} )| = \sum_{n=1}^{\oo} | z_n| | (b_n - b_{n+1} )| \leqslant  C \sum_{n=1}^{\oo} | (b_n - b_{n+1} )| =\]
    \[= C \lim_{N\to \oo} (b_1 - b_{N+1}) = C b_1 < + \oo\]
    Zatem szereg $\mathlarger{\sum}_{n=1}^{\oo} z_n ( b_n - b_{n+1})$ jest zbieżny bezwzględnie, a więc i zbieżny.
    Z tego i z równości:
    \[\sum_{n=1}^{\oo} a_n b_n = \sum_{n=1}^{\oo} z_n ( b_n - b_{n+1}) + \lim _{n \to \oo} z_n b_n, \]
    Wynika, że szereg $\mathlarger{\sum}_{n=1}^{\oo} a_n b_n$ musi być zbieżny.
\end{dow}
Kolejnym kryterium zbieżności dla szeregów o wyrazach dowolnych jest prosto wynikające z kryterium Dirichleta tzw.:
\begin{twier}{Kryterium Abela}
    Jeśli:
    \begin{itemize}
        \item mamy ciągi:
              \[(a_n)_{n\in \N}, \quad a_n \in \C \]
              \[(b_n)_{n\in \N}, \quad b_n \in \R\]
        \item $b_n$ jest monotoniczny i ograniczony,
        \item $\mathlarger{\sum}_{n=1}^{\oo} a_n$ jest zbieżny.
    \end{itemize}
    Wtedy szereg $\mathlarger{\sum}_{n=1}^{\oo} a_n b_n$ jest zbieżny.
\end{twier}
\begin{dow}
    Oczywiście $b = \lim _{n \to \oo} b_n$ istnieje. Wtedy:
    \[\sum_{n=1}^{N} a_n b_n =  \sum_{n=1}^{N} a_n (b_n - b) + b \sum_{n=1}^{N} a_n.\]
    Szereg $b \sum_{n=1}^{\oo} a_n$ jest oczywiście zbieżny na mocy założenia. Za to $\sum_{n=1}^{\oo} a_n (b_n - b)$ jest zbieżne na mocy kryterium Dirichleta, zauważmy bowiem, że:
    \[\sum_{n=1}^{N} a_n (b_n - b) = \text{sign}(b_n - b) \sum_{n=1}^{N} a_n |b_n - b|,\]
    ale $|b_n - b| \searrow 0$ na mocy założenia, a sumy częściowe $|\sum_{n=1}^{N} a_n|$ muszą być ograniczone, gdyż są zbieżne. Warunki kryterium Dirichleta są więc spełnione. Ostatecznie:
    \[\sum_{n=1}^{\oo} a_n b_n = \lim_{N \to \oo} \sum_{n=1}^{N} a_n b_n =  \sum_{n=1}^{\oo} a_n (b_n - b) + b \sum_{n=1}^{\oo} a_n.\]
    Wszystkie składniki po prawej są zbieżne, zatem szereg $\mathlarger{\sum}_{n=1}^{\oo} a_n b_n$ także musi być zbieżny.
\end{dow}
\paragraph{Grupowanie składników} Zajmijmy się teraz kwestią grupowania składników w szeregach o wyrazach dowolnych i kiedy taka operacja nie zmienia wartości szeregu. Prowadźmy jednak trochę notacji.
\begin{defr}{Rozbicie na wyrazy dodatnie}
    Dla szeregu $\mathlarger{\sum}_{n=1}^{\oo} z_n$ definiujemy następujące ciągi:
    \begin{itemize}
        \item \[a_n = \text{Re}(z_n)\]
        \item \[b_n = \text{Im}(z_n)\]
        \item \[a_n^+ = \max\{0, a_n\}\]
        \item \[a_n^- = -\min\{0, a_n\}\]
        \item \[b_n^+ = \max\{0, b_n\}\]
        \item \[b_n^- = -\min\{0, b_n\}\]
    \end{itemize}
    Jasnym jest, że:
    \[z_n = a_n^+ - a_n^- + i b_n^+ - i b_n^-\]
    oraz:
    \[0 \leqslant a_n^+, \,a_n^-, \, b_n^+,\, b_n^- \leqslant |z_n|. \]
\end{defr}
Z ostatniej nierówności wynika (kryterium porównawcze), że jeśli szereg $\mathlarger{\sum}_{n=1}^{\oo} |z_n|$ jest zbieżny, to i szeregi $\mathlarger{\sum}_{n=1}^{\oo} a_n^{\pm}$, $\mathlarger{\sum}_{n=1}^{\oo} b_n^{\pm}$ muszą być zbieżne.
Wtedy:
\[\sum_{n=1}^{\oo} z_n = \sum_{n=1}^{\oo} a_n^+ - \sum_{n=1}^{\oo} a_n^- + i\sum_{n=1}^{\oo} b_n^+ -i\sum_{n=1}^{\oo}b_n^-,\]
jako że:
\[\sum_{n=1}^{N} z_n = \sum_{n=1}^{N} a_n^+ - \sum_{n=1}^{N} a_n^- + i\sum_{n=1}^{N} b_n^+ -i\sum_{n=1}^{N}b_n^-,\]
Z tego wynika następujący wniosek:
\begin{twier}{Grupowanie szeregów zbieżnych bezwzględnie}
    Niech szereg $\mathlarger{\sum}_{n=1}^{\oo} z_n$ będzie zbieżny bezwzględnie. Wtedy:
    \begin{enumerate}
        \item Jeśli $\sigma : \N \rightarrow \N$ jest bijekcją (permutacją indeksów), to:
              \[\sum_{n=1}^{\oo} z_{\sigma(n)} = \sum_{n=1}^{\oo} z_{n}.\]
              Oznacza to, że zmiana kolejności sumowania nie wpływa na wynik.
        \item Jeśli:
              \[\N = \bigsqcup_{\substack{j=1}}^{\oo} A_j\]
              i:
              \[S_j = \sum_{n \in A_j} z_n,\]
              to $\mathlarger{\sum}_{j=1}^{\oo} S_j$ jest zbieżny bezwzględnie i:
              \[\sum_{j=0}^{\oo}S_j  = \sum_{n=1}^{\oo} z_n.  \]
    \end{enumerate}
    Są to prawa grupowania dla szeregów o wyrazach dowolnych, analogiczne do tych, które zachodzą dla szeregów o wyrazach dodatnich.
\end{twier}
\begin{dow}
    \begin{enumerate}
        \item Mamy następujące równości:
              \[\sum_{n=1}^{\oo} z_{\sigma(n)} =  \sum_{n=1}^{\oo} a_{\sigma(n)}^+ - \sum_{n=1}^{\oo} a_{\sigma(n)}^- + i\sum_{n=1}^{\oo} b_{\sigma(n)}^+ -i\sum_{n=1}^{\oo}b_{\sigma(n)}^- =\]
              \[= \sum_{n=1}^{\oo} a_n^+ - \sum_{n=1}^{\oo} a_n^- + i\sum_{n=1}^{\oo} b_n^+ -i\sum_{n=1}^{\oo}b_n^- = \sum_{n=1}^{\oo} z_n\]
              W 2 równości korzystamy z analogicznego prawa dla zbieżnych szeregów dodatnich.
              Oczywiście, jako że $\mathlarger{\sum}_{n=1}^{\oo} z_n$ jest zbieżny bezwzględnie, to mamy:
              \[\sum_{n=1}^{\oo} |z_n| = \sum_{n=1}^{\oo} |z_{\sigma(n)}|,\]
              więc permutacja szeregu zbieżnego bezwzględnie jest także zbieżna bezwzględnie.

        \item Mamy:
              \[\sum_{j=1}^{\oo} |S_j| = \sum_j^{\oo} \Big| \sum_{n \in A_j} z_n \Big| \leqslant \sum_{j=1}^{\oo} \sum_{n \in A_j} |z_n| =\sum_{n=1}^{\oo} |z_n| < + \oo. \]
              Widzimy więc, że $\mathlarger{\sum}_{j=1}^{\oo} S_j$ jest także zbieżny bezwzględnie.
              Jako że $\mathlarger{\sum}_{n=1}^{\oo} z_n$ jest zbieżny bezwzględnie, więc tym bardziej jego podszeregi muszą być zbieżne bezwzględnie. Możemy więc zapisać:
              \[\sum_{j=1}^{N} S_j = \sum_j^{N} \sum_{n \in A_j} z_n = \sum_{j=1}^N \Big(\sum_{n \in A_j} a_n^+ - \sum_{n \in A_j} a_n^- + i\sum_{n \in A_j} b_n^+ -i\sum_{n \in A_j} b_n^- \Big) =\]
              \[\sum_{j=1}^N\sum_{n \in A_j} a_n^+ - \sum_{j=1}^N\sum_{n \in A_j} a_n^- + i\sum_{j=1}^N\sum_{n \in A_j} b_n^+ -i\sum_{j=1}^N\sum_{n \in A_j} b_n^-.\]
              Każdy z szeregów po prawej stronie jest szeregiem zbieżnym o wyrazach dodatnich. Biorąc granicę $N \to \oo$ otrzymamy więc:
              \[\sum_{j=1}^{\oo} S_j = \sum_{j=1}^\oo\sum_{n \in A_j} a_n^+ - \sum_{j=1}^\oo\sum_{n \in A_j} a_n^- + i\sum_{j=1}^\oo\sum_{n \in A_j} b_n^+ -i\sum_{j=1}^\oo\sum_{n \in A_j} b_n^- = \sum_{n=1}^\oo z_n,\]
              co wynika z odpowiednich twierdzeń dla szeregów dodatnich.
    \end{enumerate}
\end{dow}

\newpage
\section{Przestrzenie metryczne}
\subsection{Podstawowe definicje. Otwartość i domkniętość}
\paragraph{} Zmienimy teraz temat, odchodząc od analizy zbieżności w liczbach zespolonych i rozpoczniemy rozważania o temacie znacznie bardziej ogólnym, mianowicie o przestrzeniach z metrykami, będących uogólnieniem znanego pojęcia odległości w $\C$.
\begin{defr}{Metryka}

    Ustalmy $X$ będące dowolnym niepustym zbiorem. Funkcję:
    \[d: X \times X \to [0; +\oo [\]
    nazywamy \textbf{metryką}, jeśli spełnia następujące aksjomaty:
    \begin{enumerate}

        \item \[\forall_{x, y \in X}: d(x,y) = d(y,x),\]
        \item \[\forall_{x, y \in X}: d(x,y) = 0 \iff x = y,\]
        \item \[\forall_{x, y, z \in X}: d(x,z) \leqslant d(x, y) + d(y, z).\]
    \end{enumerate}
\end{defr}
\paragraph{} O metryce myśleć można, jako o funkcji zwracającej "odległość" między dwoma elementami w zbiorze $X$. Podane aksjomaty zapewniają, że nasza metryka spełniać będzie "zdroworozsądkowe" własności odległości.
\textbf{1.} nakłada warunek symetryczności na metrykę - odległość z $x$ do $y$ musi być równa odległości z $y$ do $x$. \textbf{2.} normalizuje metrykę, mówiąc, że punkt jest odległy o 0 od samego siebie i \textbf{tylko} od samego siebie.
\textbf{3.} to tak zwana \textbf{nierówność trójkąta} - dodając na drodze między dwoma punktami trzeci punkt nie można odległości skrócić.

\begin{defr}{Przestrzeń metryczna}
    Parę $(X, d)$ - gdzie $X$ to niepusty zbiór, a $d: X \times X \to [0; +\oo [$ to metryka - nazywamy \textbf{przestrzenią metryczną}.
\end{defr}
\paragraph{Przykłady} Pokażemy parę przykładów przestrzeni metrycznych, aby dać pojęcie, jak mogą one wyglądać.
\begin{itemize}
    \item $X = \R$ $d(x, y) = |x - y|$ - jest to odległość między dwiema liczbami rzeczywistymi, z której korzystaliśmy np. przy definicji granicy ciągu.
    \item $X = \R^\nu$, gdzie $\nu \in \N$ jest wymiarem przestrzeni,
          \[d_p(x, y) = \Big(\sum_{n=1}^\nu |x_n - y_n| ^p\Big)^{\frac{1}{p}}.\]
          Podstawiając za $p$ różne wartości możemy otrzymać wiele alternatywnych metryk. \\
          Np. dla $p = 1$ otrzymujemy tzw. \textbf{metrykę Manhattanu}:
          \[ d_1(x, y) = \sum_{n=1}^\nu |x_n - y_n|.\]
          Nazwa pochodzi od tego, że jest to odległość jaką trzeba pokonać między dwoma punktami, mogąc przemieszczać się tylko równolegle do osi współrzędnych - tak jak na Manhattanie, gdzie ulice są do się prostopadłe. \\
          Dla $p = 2$ otrzymujemy znaną \textbf{odległość Euklidesową}:
          \[d_2(x, y) = \sqrt{\sum_{n=1}^\nu (x_n - y_n) ^2}.\]
    \item $X = \R^\nu$,
          \[d_\oo (x, y) = \max_{1 \leqslant n \leqslant \nu}{|x_n - y_n|}.\]
          Jest to tak zwana \textbf{metryka maximum}. Indeks $\oo$ wziął się z faktu, że o $d_\oo$ myśleć można o jako o granicy $d_p$ dla $p \to \oo$, mamy bowiem:
          \[\lim_{p \to \oo} \Big( \sum_{n=1}^{\nu} a_n^p \Big)^{\frac{1}{p}} = \max_{1 \leqslant n \leqslant \nu}{a_n}.\]
    \item Dla dowolnego zbioru $X$ definiujemy \textbf{metrykę dyskretną}:
          \begin{equation}
              d(x, y) = \delta_{x, y} = \begin{cases}
                  1 & x = y    \\
                  0 & x \neq y \\
              \end{cases}.
          \end{equation}
    \item Niech $X$ będzie zbiorem funkcji klasy $C^0$ (tj, funkcji ciągłych) z $[0; 1]$ na $\C$. Wtedy za odległość między dwiema funkcjami przyjąć możemy:
          \begin{equation}
              d(f, g) = \sup_{x \in [0; 1]} {|f(x) - g(x)|}.
          \end{equation}
    \item Dla $X$ takiego samego jak w poprzednim punkcie można określić także:
          \begin{equation}
              d_p(f, g) = \Big( \int_{0}^{1} |f(x) - g(x)|^p \Big)^{\frac{1}{p}}.
          \end{equation}
\end{itemize}
\begin{defr}{Zbieżność w przestrzeni metrycznej}
    Niech $(X, d)$ będzie przestrzenią metryczną i niech $(x_n)_{n \in \N}$ będzie ciągiem elementów z $X$. Powiemy, że $(x_n)_{n \in \N}$ jest zbieżny do $g \in X$, jeśli:
    \begin{equation}
        \forall_{\varepsilon > 0} \exists_{N_\varepsilon \in \N}\forall_{n \geqslant N_\epsilon}: d(x_n, g) < \varepsilon,
    \end{equation}
    co możemy alternatywnie zapisać, jako:
    \begin{equation}
        \lim_{n \to \oo} d(x_n, g) = 0.
    \end{equation}
\end{defr}
\paragraph{} Oczywiście jeśli $x_n \to g$ i $x_n \to g'$, to $g = g'$ - co wynika z faktu, że jedynym elementem odległym o 0 od $g$ jest $g$. Ponadto jeśli $x_n \to g$ to każdy podciąg $(x_{n_k})_{k \in \N}$ także dąży do $g$: $x_{n_k} \xrightarrow{k \to \oo} g$.


\begin{defr}{Równoważność metryk}
    Niech $d$ i $\delta$ będą metrykami na $X$. Powiemy, że $d$ i $\delta$ są równoważne, jeśli:
    \begin{equation}
        \exists_{C_1, C_2 \in \R}: \forall_{x, y \in X}: C_1 d(x, y) \leqslant \delta (x, y) \leqslant C_2 d(x, y).
    \end{equation}
    Łatwo widać, że jest to relacja symetryczna, przechodnia i zwrotna.

\end{defr}

\paragraph{Przykład} Metryk $d_p$ i $d_\oo$ na $\R^\nu$ są równoważne, albowiem:
\begin{equation}
    d_oo(x, y) = \max_{i} |x_i - y_i| \leqslant d_p(x, y) = \Big(\sum_{n=1}^\nu |x_n - y_n| ^p\Big)^{\frac{1}{p}}.
\end{equation}
Ponadto:
\begin{equation}
    d_p(x, y) ^p \leqslant \nu d_\oo(x, y)^p \implies d_p(x, y) \leqslant \nu^\frac{1}{p} d_\oo(x, y)
\end{equation}
Wystarczy więc wziąć $C_1 = 1$ i $C_2 = \nu ^\frac{1}{p}$.

\begin{obs}{}
    Jeśli $d$ i $\delta$ są równoważne, to $x_n \to g$ w $(X, d)$ $\iff$ $x_n \to g$ w $(X, \delta)$.
\end{obs}

\begin{defr}{Warunek Cauchy'ego}
    Powiemy, że ciąg $\ciag{x}$ spełnia \textbf{warunek Cauchy'ego} (i.e. że jest \textbf{ciągiem Cauchy'ego}), jeśli:
    \begin{equation}
        \forall_{\eps > 0} \exists_{N_\eps \in \N} \forall (m,n \geqslant N_\eps): d(x_m, x_n) \leqslant \eps.
    \end{equation}
\end{defr}

\begin{defr}{Przestrzeń zupełna}
    Powiemy, że przestrzeń $(X,d)$ jest \textbf{zupełna} jeśli każdy ciąg Cauchy'ego w tej przestrzeni jest zbieżny.
\end{defr}

\paragraph{} Zauważmy, że $()\R, d_2)$ (czyli liczby rzeczywiste ze standardową metryką Euklidesową) to przestrzeń zupełna, co udowodniliśmy. $(\Q, d_2)$ nie jest przestrzenią zupełną (gdyż np. ciąg kolejnych przybliżeń dziesiętnych $\sqrt[]{2}$ spełnia warunek Cauchy'ego, jednak nie ma w $\Q$ granicy).\

\begin{defr}{Kule}
    Niech $(X, d)$ - p-ń metryczna. Wtedy \textbf{kulą (otwartą)} o środku $x_0 \in X$ i promieniu $r \in \R_+$ nazwiemy zbiór:
    \begin{equation}
        \ball{x_0}{r} = \{ x\in X \, | \, d(x, x_0) < r\}.
    \end{equation}
\end{defr}

\begin{defr}{Punkty wewnętrzne}
    Niech $(X, d)$ - p-ń metryczna, $A \subseteq X$ i $a \in A$. Powiemy, że $a$ jest \textbf{punktem wewnętrznym} $A$, jeśli:
    \begin{equation}
        \exists_{r>0}: \ball{a}{r} \subseteq A.
    \end{equation}
\end{defr}

\paragraph{} Widzimy więc, że jeśli $a$ jest punktem wewnętrznym $A$, to znajduje się w $A$ razem z pewną kulą wokół siebie - można potocznie sobie więc wyobrazić, że $a$ nie może być na "brzegu" $A$.

\begin{defr}{Zbiór otwarty}
    Powiemy, że $A$ jest \textbf{otwarty} (w ustalonej p-ń metrycznej $(X,d)$ ), jeśli każdy jego punkt jest punktem wewnętrznym:
    \begin{equation}
        \forall_{a \in A} \exists_{r>0}: \ball{a}{r} \subseteq A.
    \end{equation}
    \\
    Dla ustalonej przestrzenii metrycznej zdefiniujemy $\Tau$ jako rodzinę wszystkich zbiorów otwartych w tej przestrzenii. Wprowadzimy też zapis:
    \begin{equation}
        U \subotw X,
    \end{equation}
    jeśli $U$ jest otwartym podzbiorem $X$ (i.e. jeśli $U \in \Tau$).
\end{defr}


\begin{twier}{Właności $\Tau$}
    \begin{enumerate}
        \item $\emptyset \in \Tau$ oraz $X \in \Tau$.
        \item $U, V \in \Tau \implies U \cap V \in \Tau$. Inaczej mówiąc (rozszerzywszy łątwo twierdzenie za pomocą indukcji), skończony iloczyn zbiorów otwartych jest otwarty.
        \item $\{U_i\}_{i \in I}$ - rodzina zbiorów otwartych. Wtedy \begin{equation}
                  \bigcup_{i \in I} U_i \in \Tau.
              \end{equation} Zauważmy, że nie zakładamy, że jest to skończona (ani nawet przeliczalna) rodzina.

    \end{enumerate}
\end{twier}

\begin{dow}
    \begin{enumerate}
        \item Prawdziwym jest zdanie, że dla każego elementu należącego do $\emptyset$ jest on punktem wewnętrzym. Ponadto: \begin{equation}
                  \forall_{x \in X,\, r > 0} \ball{x}{r} \subseteq X,
              \end{equation}
              co wynika za samej definicji Ball.

        \item Skoro $U, V$ są otwarte, to dla każdego $x \in U \cap V$ istnieją $r_1$ i $r_2$ takie, że:\begin{equation*}
                  \ball{x}{r_1} \in U, \quad \ball{x}{r_2} \in V.
              \end{equation*} Wtedy jednak: \begin{equation*}
                  \ball{x}{\min\{r_1, r_2\}} \in U, \quad \ball{x}{\min\{r_1, r_2\}} \in V.
              \end{equation*} Wystarczy więc wziąć $r = \min\{r_1, r_2\}$

        \item Dla każdego $x \in \bigcup_{i \in I} U_i$ mamy $i \in I$ takie, że $x \in U_i$. Wtedy jednak istnieje takie $r$, że $\ball{x}{r} \in U_i$, przeto $\ball{x}{r} \in \bigcup_{i \in I} U_i$.

    \end{enumerate}

\end{dow}

\begin{twier}{}
    Kula otwarta jest otwarta.
\end{twier}

\begin{dow}
    Istotnie, jeśli niech $y \in \ball{x}{r}$. Wtedy $d(x, y) < r$. Zauważmy, że $\ball{y}{r - d(x, y)} \subseteq \ball{x}{r}$, bowiem jeśli $z \in \ball{y}{r - d(x, y)}$, to $d(z, y) < r - d(x, y)$, zatem: \begin{equation*}
        d(z, x) < d(z, y) + d(y, z) < r,
    \end{equation*} czyli $z \in \ball{x}{r}$.
\end{dow}



\begin{defr}{Punkt skupienia}
    Niech $(X, d)$ - p-ń metryczna i $A \subseteq X$, wtedy punkt $x \in X$ nazwiemy \textbf{punktem skupienia} $A$, jeśli: \begin{equation}
        \forall_{r>0}: \ball{x}{r} \cap A \neq \emptyset.
    \end{equation}
\end{defr}

\begin{defr}{Zbiór domknięty}
    Zbiór $A$ nazwiemy \textbf{domkniętym} jeśli zawiera wszystkie swoje punkty skupienia. To znaczy: \begin{equation}
        \forall_{x \in X} [\forall_{r>0}: \ball{x}{r} \cap A \neq \emptyset \implies x \in A].
    \end{equation}
    \\
    Podobnie jak $\Tau$, zdefiniujemy $\Fau$ jako rodzinę wszystkich zbiorów domkniętych w danej przestrzenii. Będziemy również pisać:
    \begin{equation}
        U \subdomk X,
    \end{equation}
    jeśli $U$ jest domkniętym podzbiorem $X$.
\end{defr}

\begin{defr}{Kula domknięta}
    \textbf{Kulą domkniętą} o środku $x_0 \in X$ i promieniu $r \in \R_+$ nazwiemy zbiór: \begin{equation}
        \dball{x}{r} = \{ x\in X \, | \, d(x, x_0) \leqslant r\}.
    \end{equation}
\end{defr}

\begin{twier}{}
    Kula domknięta jest domknięta.
\end{twier}

\begin{dow}
    Niech $y \notin \dball{x}{r}$. Wtedy $d(y, x) > r$. Weźmy $R = d(y, x) - r > 0$. Wtedy $\ball{y}{R} \cap \dball{x}{r} = \emptyset$. Albowiem, jeśli $z \in \ball{y}{R}$, to $d(z, y) <  d(y, x) - r $, zatem $d(x, z) \geqslant d(x, y) - d(z, y)  > r$, czyli $z \notin \dball{x}{r}$.
\end{dow}

\begin{twier}{Dopełnienie zbioru otwartego jest domknięte}\label{twier:dopelnienie-otwartego}
    Niech $A \subseteq X$. $A$ jest otwarty wtedy i tylko wtedy, gdy $X \backslash A$ jest domknięty.
\end{twier}



\begin{dow}
    Mamy następujący ciąg równoważności:
    \begin{equation}
        (X \backslash A) \text{ - domknięty} \iff
    \end{equation}
    \begin{equation}
        \forall_{y \notin (X \backslash A)} \exists_{r > 0}: \ball{y}{r} \cap (X \backslash A) =  \emptyset \iff
    \end{equation}
    \begin{equation}
        \forall_{y \in A} \exists_{r > 0}: \ball{y}{r} \subseteq X \backslash (X \backslash A) = A \iff
    \end{equation}
    \begin{equation}
        A \text{ - otwarty}
    \end{equation}
\end{dow}

Zauważmy, że istnieją zbiory jednocześnie domknięte i otwarte. Na przykład w każdej przestrzenii metrycznej są to $\emptyset$ oraz cała przestrzeń. W $\R$ są to jedynie takie zbiory. Za to w przestrzenii dyskretnej, wszystkie zbiory mają te własność.


\begin{twier}{}
    Zbiór $A$ jest otwarty wtedy i tylko wtedy, gdy $A$ jest sumą kul.
\end{twier}
\begin{dow}
    \paragraph{$\implies$} Możemy wybrać sumę rodziny generowanej w tej sposób, że każdemu elementowi przypisujemy kulę mu odpowiadającą, która należy do $A$: \begin{equation}
        \forall_{y \in A} \exists_{r(y) >0}: \ball{y}{r(y)} \subseteq A.
    \end{equation} Zatem: \begin{equation}
        A = \bigcup_{y \in A} \{ y \} \subseteq \bigcup_{y \in A} \ball{y}{r(y)}
    \end{equation} oraz: \begin{equation}
        \bigcup_{y \in A} \ball{y}{r(y)} \subseteq A,
    \end{equation} więc:\begin{equation}
        A = \bigcup_{y \in A} \ball{y}{r(y)}
    \end{equation}

    \paragraph{$\impliedby$} W drugą stronę sprawa jest jasna - suma każdej rodziny kul (i.e. zbiorów otwarty) będzie otwarta.
\end{dow}

\begin{obs}{na temat $\Fau$}
    \begin{enumerate}
        \item $\emptyset, X \in \Fau.$
        \item $\Fau = \{ X \backslash A \, \big| \, A \in \Tau \}.$
        \item $A, B \in \Fau \implies A \cup B \in \Fau.$ Inaczej mówiąc - skończony iloczyn zbiorów domkniętych jest domknięty.
        \item $\{V_i\}_{i \in I}$ - rodzina zbiorów domkniętych. Wtedy \begin{equation}
                  \bigcap_{i \in I} V_i \in \Tau.
              \end{equation}
    \end{enumerate}
\end{obs}
\paragraph{} Fakty te są oczywistą konsekwencją własności $\Tau$ i prawa, które mówi, że dopełnienie zbioru otwartego jest domknięte.
\paragraph{} Spójrzmy jeszcze na przykład, pokazujący, że w 3. właność ta zachodzi tylko dla sum skończonych: \begin{equation*}
    ]0, 1[ = \bigcup_{i=1}^{\oo} [\frac{1}{n}, 1-\frac{1}{n}].
\end{equation*}

\begin{defr}{Wnętrze i domknięcie}
    Niech $B \subseteq X$.\\
    \textbf{Wnętrznem} zbioru $B$, oznaczanym $B \dg$, nazywamy zbiór wszyskich punktów wewnętrznych $B$.\\
    \textbf{Domknięciem} zbioru $B$, oznaczanym $\overline{B}$, nazywamy zbiór punktów skupienia $B$. \\
    Innymi słowy: \begin{equation}
        x \in B \deg \iff \exists_r: \ball{x}{r} \subseteq B.
    \end{equation} \begin{equation}
        x \in \overline{B} \iff \forall_r: \ball{x}{r} \cap B \neq \emptyset.
    \end{equation}
\end{defr}

\begin{obs}{}
    \begin{equation}
        B \dg \subseteq B \subseteq \overline{B}
    \end{equation}
\end{obs}

\begin{obs}{}
    \begin{itemize}
        \item $B$ - otwarty $\iff$ $B \dg = B.$
        \item $B$ - domknięty $\iff$ $\overline{B} = B.$
    \end{itemize}
\end{obs}

\begin{twier}{}
    \begin{equation}
        X \backslash \overline{B} = (X \backslash B) \dg
    \end{equation} lub \begin{equation}
        \overline{B} = X \backslash (X \backslash B) \dg.
    \end{equation}
\end{twier}

\begin{dow}
    $x \in X \backslash \overline{B} \iff x$ nie jest punktem skupienia $B \iff \exists_r \ball{x}{r} \cap B = \emptyset \iff \exists_r \ball{x}{r} \cap B \subseteq X \backslash B \iff x \in (X \backslash B) \dg$.
\end{dow}


Powiemy teraz o bardzo ważnym twierdzeniu pozwalającym utożsamić punkty skupienia z granicami ciągów ze zbioru.
\begin{twier}{}
    Niech $B \subseteq X$. Wtedy domknięcie $B$ to zbiór granic ciągów z $B$, tj.: \begin{equation}
        \overline{B} = \{ \lim_{n\to \oo} x_n \,  \big| \, \ciag{x} \wedge \forall_n x_n \in B \}.
    \end{equation}
\end{twier}

\begin{dow}
    \paragraph{$\supseteq$} Dla każdego $x \in \overline{B}$ zdefiniujmy ciąg $\ciag{x}$ następująco - niech $x_n$ to dowolny element należący do $\ball{x}{\frac{1}{n}} \cap B$ (jest to zbiór zawsze niepusty na mocy tego, że $x$ jest punktem skupienia $B$). Oczywiście $ x_n \to x$, gdyż $d(x_n, x) < \frac{1}{n} \to 0$.
    \paragraph{$\subseteq$} Odwrotnie, jeśli $x_n \to x$ i $\forall_n : x_n \in B$, to: \begin{equation}
        \forall_{\eps > 0} \exists_{N_\eps} \forall_{n \geqslant N_\eps}: B \ni x_n \in \ball{x}{\eps} \implies \forall{\eps >0} \ball{x}{\eps} \cap B \neq \emptyset,
    \end{equation} skąd mamy $x \in \overline{B}$.
\end{dow}

\begin{defr}{Ograniczoność}
    Powiemy, że $B \subseteq X$ jest \textbf{ograniczony}, jeśli: \begin{equation}
        \exists_{x \in X, r >0}: B \subseteq \ball{x}{r}.
    \end{equation}
\end{defr}

\begin{defr}{Gęstość}
    Powiemy, że $A \subseteq X$ jest \textbf{gęsty} w $B \subseteq X$, jeśli: \begin{equation}
        B \subseteq \overline{A}.
    \end{equation}
    Uwaga! Mówiąc po prostu, że zbiór jest gęsty, mamy na myśli, że jest gęsty w $X$.
\end{defr}
\paragraph*{Przykład} $\Q$ jest gęsty w $\R$.

\begin{defr}{Otoczenie}
    $A \subseteq X$ jest otoczeniem punktu $x \in X$, jeśli: \begin{equation}
        \exists_{U \subotw X}: x \in U \subseteq A.
    \end{equation} Waruneke ten jest równoważny temu, żę $x \in A \dg$. \\
    Dla danego punktu rodzinę wszyskich jego otoczeń oznaczać będziemy $\Nau (x)$: \begin{equation}
        \Nau (x) = \{ A \in \mathcal{P}(X) = 2^X \, \big| \, x \in A\dg \}.
    \end{equation}
\end{defr}

\begin{twier}{}
    Niech $\ciag{x}$ będzie ciągiem elementów $X$. Wtedy:\begin{equation}
        \lim_{n \to \oo} x_n = x \iff \forall_{A \in \Nau(x)} \exists_{N_A \in \N} \forall_{n \geqslant N_A}: x_n \in A.
    \end{equation}
\end{twier}

\begin{dow}
    \paragraph{$\impliedby$} Weźmy $A_\eps  \ball{x}{\eps}$. Wtedy:\begin{equation}
        \forall_{\eps >0} \exists_{N_\eps} \forall_{n \geqslant N_\eps}: x_n \in \ball{x}{\eps} \implies d(x_n, x) < \eps.
    \end{equation}
    \paragraph{$implies$} $x_n$ zbiega do $x$, zatem: \begin{equation*}
        \forall_{\eps >0} \exists_{N_\eps} \forall_{n \geqslant N_\eps}: d(x_n, x) < \eps.
    \end{equation*} Mamy też: \begin{equation*}
        A \in \Nau(x) \implies \exists_\eps: \ball{x}{\eps} \subseteq A.
    \end{equation*} Wystarczy więc wziąć $N_A = N_\eps$. Wtedy będzie: \begin{equation}
        \forall_{\eps >0} \exists_{N_\eps} \forall_{n \geqslant N_\eps}: (d(x_n, x) < \eps \wedge \ball{x}{\eps} \subseteq A) \implies x_n \in A.
    \end{equation}
\end{dow}

\paragraph*{} Twierdzenie to mówi nie mniej, nie więcej niż to, że jeśli ciąg zbiega do jakiegoś punktu, to w każdym otoczeniu tego punktu znajdą się prawie wszystkie wyrazy tego ciągu.

\begin{twier}{}
    Niech $(X, d)$ - p-ń. metryczna, niech $Y \subseteq X$ i niech $d_y = d | _{Y \times Y}$ będzie obcięciem $d$ do $Y$. Wtedy:
    \begin{enumerate}
        \item $A \subseteq Y$ jest otwarty w $(Y, d_y)$ $\iff$ $\exists_{U \subotw X}: A = Y \cap U$.
        \item $A \subseteq Y$ jest domknięty w $(Y, d_y)$ $\iff$ $\exists_{F \subdomk X}: A = Y \cap F$.
    \end{enumerate}
\end{twier}
Twierdzenie to daje nam pojęcie o tym, jak wyglądają zbiory otwarte i domknięte w jakimś zawężeniu metryki - mianowicie są to zawężenia odpowiednich zbiorów otwartych i domkniętych.

\begin{dow}
    Zaczniemy od dowodu 2.
    \paragraph{$\implies$} Niech $A$ domknięty w $Y$ i niech $F$ będzie domknięciem $A$ w $X$ (więc $F$ musi być domknięty w $X$). Wtedy: \begin{equation*}
        F = \{ \lim_{n \to \oo} a_n \in X \, \big| \, a_n \in A \},
    \end{equation*} za to:\begin{equation*}
        A = \{ \lim_{n \to \oo} a_n \in Y \, \big| \, a_n \in A \},
    \end{equation*}gdyż $A$ jest domknięty w $Y$. Widzimy z tych definicji łatwo, że $A = F \cap Y$.
    \paragraph{$\impliedby$} W drugą stronę, niech $A = Y \cap F$ i $F$ będzie domknięty w $X$. Niech $\ciag{a}$ będzie ciągiem w $A$ (więc i w $F$), zbiegającym do $g$. Wtedy $g \in F$. Zatem jeśli $g \in Y$, to $g \in A$, z konstrukcji $A$. Pokazaliśmy więc, że każdy ciąg z $A$ zbieżny w $Y$ ma granicę w samym $A$, zatem $A$ jest domknięty.
    \paragraph{$2. \implies 1.$} Pokażemy teraz, jak z 2. wynika 1. Mamy bowiem ciąg następujących równoważności: \\
    $A \subseteq Y$ jest otwarty w $Y$ $\iff$ $Y \backslash A$ jest domknięty w $Y$ $\iff$ $\exists_{F \subdomk X}: Y \backslash A = Y \cap F$ $\iff$ $ \exists_{F \subdomk X}: A = Y \cap (X \backslash F)$ $\iff$ $\exists_{U \subotw X}: A = Y \cap U$.\\
    Przedostatnia równoważność, wynika z faktu, że jeśli $Y \backslash A = Y \cap F$, to $A = Y \backslash (Y \cap F) = Y \backslash F =Y \cap (X \backslash F)$ (pamiętamy, że $Y\subseteq X)$.
    Ostatnia równoważność jest konsekwencją tego, że dopełnienie zbioru domkniętego jest otwarte (Tw. \ref{twier:dopelnienie-otwartego}).
\end{dow}

\subsection{Przekształcenia ciągłe}

Zaczniemy rozważać teraz przekształcenia (i.e. funkcje) pomiędzy przestrzeniami metrycznymi i wyróżnimy wśród nich szczgólnie ważną klasę funkcji ciągłych.

\begin{twier}{Warunki ciągłośći} \label{twier:warunki-ciaglosci}
    Niech $(X, d)$ i $(Y, \delta)$ będą przestrzeniami metrycznymi, a $\varphi : X \to Y$ przekształceniem między nimi. Ustalmy też $x_0 \in X$. Wtedy następujące warunki są równoważne:
    \begin{enumerate}
        \item \begin{equation}
                  \forall_{\eps > 0} \exists_{\lambda_\eps >0} \forall_{x \in X}: d(x, x_0) < \lambda_\eps \implies \delta(\varphi(x), \varphi(x_0)) < \eps.
              \end{equation}
        \item \begin{equation}
                  \forall_{U \in \Nau(\varphi(x_0))} \exists_{V \in \Nau(x_0)}: \varphi(V) \subseteq U.
              \end{equation}
        \item \begin{equation}
                  \forall_{U \in \Nau(\varphi(x_0))}: \varphi^{-1}(U) \in \Nau(x_0).
              \end{equation} (Przypominamy, że $\varphi^{-1}(U)$ to w przeciwobraz $U$).
        \item Jeśli $x_n \to x_0$ w $(X, d)$, to $\varphi(x_n) \to \varphi(x_0)$ w $(Y, \delta)$.
    \end{enumerate}
\end{twier}

\begin{dow}
    \paragraph{$1. \implies 2.$}
    Dla dowolnego $U \in \Nau(\varphi(x_0))$ istnieje $\eps > 0$: \begin{equation*}
        \ball{\varphi(x_0)}{\eps} \subseteq U.
    \end{equation*} Z 1. weźmy więc $\lambda_\eps$ spełniające podaną implikację i $V = \ball{x_0}{\lambda_\eps}$. Wtedy: \begin{equation*}
        \varphi( x \in V) \in \ball{\varphi(x_0)}{\eps},
    \end{equation*} zatem: \begin{equation*}
        \varphi(V) \subseteq \ball{\varphi(x_0)}{\eps} \subseteq U.
    \end{equation*}

    \paragraph{$2. \implies 3.$}
    Weźmy dowolne $U \in \Nau(\varphi(x_0))$. Wtedy: \begin{equation*}
        \exists_{V \in \Nau(x_0)}: \varphi(V) \subseteq U,
    \end{equation*} i.e. $V \subseteq \varphi^{-1}(U)$. Ale skoro $V \in \Nau(x_0)$, to tym bardziej $\varphi^{-1}(U) \in \Nau(x_0)$, jako rozszerzenie $V$.

    \paragraph{$3. \implies 4.$}
    Weźmy ciąg $\ciag{x}$ zbieżny do $x_0$. Niech $U \in \Nau(\varphi(x_0))$, skąd $\varphi^{-1}(U) \in \Nau(x_0)$. Zatem: \begin{equation*}
        \exists_{N_U} \forall_{n \geqslant N_U}: x_n \in \varphi^{-1}(U),
    \end{equation*} z twierdzenia udowodnionego wcześniej. %NR%
    Ergo: \begin{equation*}
        \forall_{U \in \Nau(\varphi(x_0))} \exists_{N_U} \forall_{n \geqslant N_U}: \varphi(x_n) \in U.
    \end{equation*} Zatem, z tego samego twierdzenia: \begin{equation*}
        \varphi(x_n) \to \varphi(x_0).
    \end{equation*}

    \paragraph{$4. \implies 1.$} Dowodzimy nie wprost, że $\neg 1. \implies \neg 4.$. Zaprzeczeniem $1.$ jest: \begin{equation*}
        \exists_{\eps > 0} \forall_\lambda \exists_{x_\lambda \in X}: d(x_\lambda, x_0) < \lambda \wedge \delta( \varphi(x_\lambda, x_0)) \geqslant \eps.
    \end{equation*} Weźmy $\lambda_n = \frac{1}{n}$ i na tej podstawie skonstruujmy ciąg $x_n$ odpowiadjących $x_\lambda$ w powyższym stwierdzeniu dla odpowiednich $\lambda$. Wtedy musi być, że $x_n \to x_0$, gdyż $d(x_n, x_0) < \frac{1}{n}$, ale $\varphi(x_n) \nrightarrow \varphi(x_0)$, gdyż $\delta( \varphi(x_n), \varphi(x_0)) \geqslant \eps \, \forall_n$. Zatem zachodzi $\neg 4.$.

\end{dow}


    \begin{defr}{Odwzorowanie ciągłe}\label{defr:ciaglosc}
        Odwzorowanie spełniające warunki 1. - 4. Twierdzenia \ref{twier:warunki-ciaglosci} dla punktu $x_0 \in X$ nazywamy \textbf{ciągłym} w $x_0$ (w p-ń. $(X, d)$). Jeśli odwzorowanie jest ciągłe w każdym punkcie przestrzenii, nazywamy je po prostu ciągłym.
    \end{defr}
        Czasami, mając na myśli warunek 1., mówimy o tzw. zbieżności według Cauchy'ego, zaś o warunku 4. mówimy, jako zbieżności według Heinego.

    \begin{twier}{Złożenie odwzorowań ciągłych jest ciągłe}
        Jeśli $\varphi: X \to Y$ jest ciągłe w $x_0 \in X$ i $\Psi: Y \to Z$ jest ciągłe w $\varphi(x_0)$, to $\Psi \circ \varphi: X \to Z$ jest ciągłe w $X_0$.
    \end{twier}

    \begin{dow}
        Korzystając np. z 4. warunku ciągłości, weźmy ciąg $x_n \to x_0$, wtedy $\varphi(x_n) \to \varphi(x_0)$ z ciągłości $\varphi$. Ponadto $\Psi(\varphi(x_n)) \to \Psi(\varphi(x_0))$ z ciągłości $\Psi$. Widzimy, więc prosto, że $\Psi \circ \varphi$ musi być ciągłe.
    \end{dow}

    \begin{twier}{Ciągłość na całej dziedzinie}
        $\varphi: X \to Y$ jest ciągłe na całym $X$ wtedy i tylko wtedy, gdy: \begin{equation}
            \forall_{U \subotw Y}: \varphi^{-1}(U) \subotw X.
        \end{equation} Inaczej mówiąc, przekształcenie jest ciągłe na całej dziedzinie wtedy i tylko wtedy, gdy przeciwobrazy zbiorów otwartych są otwarte.
    \end{twier}

    \begin{dow}
        \paragraph{$\implies$} Niech $U \subotw Y$ i $x \in \varphi^{-1}(U)$. Zatem $\varphi(x) \in U$. Skoro $U$ jest otwarty, to musi być otoczeniem $\varphi(x)$, i.e. $U \in \Nau(\varphi(x))$, zatem (warunek 3.): \begin{equation*}
            \varphi^{-1}(U) \in \Nau(x) \quad \forall_{x \in \varphi^{-1}(U)},
        \end{equation*} czyli $\varphi^{-1}(U)$ jest otwarty.

        \paragraph{$\impliedby$} Udowadniamy warunek 2.:\begin{equation*}
            \forall_{U \subotw Y}: \varphi^{-1}(U) \subotw X.
        \end{equation*} Weźmy $x \in X$ i $U \in \Nau(\varphi(x))$. Wtedy: \begin{equation*}
            \exists_{\tilde{U} \subotw X}: \varphi(x) \in \tilde{U} \subseteq U,
        \end{equation*} więc $V = \varphi^{-1}(\tilde{U})$ - otwarty. Pondato $x \in V$. Zatem $V \in \Nau(x)$, oraz $\varphi(V) \subseteq \tilde{U} \subseteq U$, więc $2.$ zachodzi.
    \end{dow}


    \begin{defr} {Ciągłość jednostajna} \label{defr:ciaglosc-jednostajna}
        $\varphi: X \to Y$ nazwiemy $\textbf{jednostajnie ciągłym}$, jeśli: \begin{equation}
            \forall_{\eps > 0} \exists_{\lambda_\eps}: \forall_{x, x' \in X}: d(x, x') \leqslant \lambda_\eps \implies \delta(\varphi(x), \varphi(x')) \leqslant \eps.
        \end{equation}
        Ważną częścią tejże definicji jest to, że wybór $\lambda$ zależy jedynie od $\eps$, nie zaś od samych punktów $x, x'$.
    \end{defr} 

    \begin{obs}{}
        Każda funkcja jednostajnie ciągła, jest ciągła.
    \end{obs}

    \begin{defr}{Przekształcenie Lipschitzowskie}
        Przekształcenie $\varphi: X \to Y$ nazwiemy \textbf{Lipschitzowskim} (i.e. spełniającym \textbf{warunek Lipschitza}), jeśli:\begin{equation}
            \exists_{L > 0} \forall_{x, x' \in X}: \delta(\varphi(x), \varphi(x')) \leqslant L \cdot d(x, x').
        \end{equation} Warunek ten mówi, że odległość między wartościami funkcji jest zawsze ograniczona przez odległość między jej argumentami (z pewną proporcjonalnością).
    \end{defr}

    \begin{obs}{}
        Funkcje Lipschitzowskie są ciągłe jednostajnie.
    \end{obs}
    \paragraph{} Istotnie - wystarczy ustalić: $\lambda_\eps = \frac{\eps}{L}$.

\subsection{Zwartość}

    Powiemy teraz o bardzo ważnym pojęciu charakteryzującym zbiory w przestrzenii metrycznej - to jest o zwartości.

    \begin{defr}{Pokrycie}
        Ustalmy $(X, d)$ - p-ń. metryczną.
        \textbf{Pokryciem} zbioru $K$ nazwiemy rodzinę zbiorów $U = \{U \}_{i \in I}$, $U_i \subseteq X$, taką że: \begin{equation}
            K \subseteq \bigcup_{i \in I} U_i.
        \end{equation}  Pokrycie jest otwarte, jeśli wszystkie $U_i$ są otwarte.
        Pokrycie $V = \{ V_i \}_{i \in I'}$ nazwiemy podpokryciem $U$, jeśli $V \subseteq U$.
    \end{defr}
    \paragraph{Przykład} Dla $X = \R$, $K = [0, 1]$, pokryciem $K$ jest rodzina: \begin{equation*}
        {U_n}_{n \in \N}: \quad U_n = [0, 1 + \frac{1}{n}].
    \end{equation*} Nie jest to pokrycie otwarte.

    \begin{defr}{Zwartość (pokryciowa)}\label{defr:zwartosc-pokryciowa}
        Zbiór $K \subseteq X$ nazwiemy \textbf{ZWARTYM}, jeśli z każdego pokrycia otwartego tegoż zbioru, można wybrać podpokrycie skończone (tj. mające skończonę liczbę elementów). Jeśli $K$ jest zwartym podzbiorem $X$, to zapiszemy $K \subseteq \subseteq X$.
    \end{defr}


    \begin{twier}{Własności zbiorów zwartych} \label{twier:zwar-domk}
        \begin{enumerate}
            \item Każdy zbiór zwarty jest ograniczony.
            \item Każdy zbiór zwarty jest domknięty.
        \end{enumerate}
    \end{twier}

    \begin{dow}
        Niech $K \subseteq \subseteq X$.
        \begin{enumerate}
            \item Weźmy dowolny $x \in X$. Wtedy $\{ \ball{x}{}n \}_{n \in \N} $ niewątpliwie jest pokryciem otwartym $K$. Skoro tak, to można zeń wybrać skończone podpokrycie. Zatem istnieje $N \in \N$, będące największym z indeksów z tegoś podpokrycia, takie że: \begin{equation*}
                K \subseteq \ball{x}{N},
            \end{equation*} jako że kolejne z tych kul zawierają w sobie poprzednie. Widzimy więc, że $K$ jest ograniczony.

            \item Dowód nie wprost. Przypuśćmy, że $\exists \, y$ będący punktem skupienia $K$, t.ż. $y \notin K$. Weźmy rodzinę zbiorów $\{ X \backslash \dball{y}{\frac{1}{n}} \}_{n \in \N}$. Niewątpliwie jest to pokrycie otwarte, gdyż wszystkie z tych zbiorów są otwarte i sumują się one do $X \backslash \{y\}$. Wybierzmy z niego więc skończone podpokrycie i niech $N$ będzie największym z indeksów w tym podpokryciu. Zauważmy, że wtedy sumujw się ono do $X \backslash \dball{y}{\frac{1}{N}}$, jako że kolejne z tych zbiorów zawierają w sobie poprzednie. Mamy więc, że $K \subseteq X \backslash \dball{y}{\frac{1}{N}}$. Wtedy jednak $ \ball{y}{\frac{1}{N}} \cap K \subseteq \dball{y}{\frac{1}{N}} \cap K = \emptyset$, co przeczy temu, że $y$ jest punktem skupienia $K$. Sprzeczność! Zatem $K$ rzeczywiście musi być domknięty.
        \end{enumerate}
    \end{dow}

    \begin{defr}{Zwartość ciągowa} \label{defr:zwartosc-ciagowa}
        Powiemy, że podzbiór $K \subseteq X$ jest \textbf{ciągowo zwarty}, jeśli każdy ciąg $\ciag{x}$ zawarty w $K$ ($x_n \in K$) zawiera podciąg zbieżny do granicy w $K$: \begin{equation}
            \forall_{\ciag{x}, x_n \in K} \exists_{k \mapsto n_k}: x_{n_k} \to g \in K.
        \end{equation}
    \end{defr}

    \begin{twier}{}\label{twier:ciag-zwarty-domk}
        Zbiór $K$ ciągowo zwarty jest także domknięty.
    \end{twier}

    \begin{dow}
        Istotnie, jeśli ciąg $\ciag{x}$ elementów z $K$ jest zbieżny, to każdy jego podciąg jest zbieżny do tej samej granicy. Ale skoro $K$ jest ciągowo zwarty, to $\ciag{x}$ ma podciąg zbieżny do elementu z $K$, który będzie wspólną granicą wszystkich podciągów. Zatem i granica $\ciag{x}$ musi leżeć w $K$. 
    \end{dow}


    \begin{defr}{$\eps$ - sieć} \label{defr:eps-siec}
        Niech $K \subseteq X$ i $\eps > 0$. Wtedy $S \subseteq X$ nazwiemy \textbf{$\eps$ - siecią dla} $K$ jeśli: \begin{equation}
            \forall_{z \in K} \exists_{s \in S}: z \in \ball{s}{\eps}.
        \end{equation} $S$ jest $\eps$ - siecią \textbf{w} jeśli jest $\eps$ - siecią dla $K$ i $S \subseteq K$.
    \end{defr}


    \begin{twier}{}\label{twier:sk-eps-siec}
        Jeśli $K$ jest ciągowo zwarty, to dla każdego $\eps > 0$ istnieje skończona $\eps$ - sieć w $K$.
    \end{twier}

    \begin{dow}
        \paragraph{Dowód nie wprost.} Przypuśćmy, że istnieje $\eps > 0$, t.ż. w $K$ nie ma skończonej $\eps$ - sieci. Skonstruujmy wtedy następująco ciąg $\ciag{x}$ i rodzinę zbiorów $\ciag{U}$:\\
         Niech $x_1 \in K$ będzie dowolne i $U_1 = \ball{x_1}{\eps}$. Oczywiście $U_1$ nie pokrywa $K$ (z założenia): $K \backslash U_1 \neq \emptyset$. Za $x_2$ bierzemy więc dowolny element w $K \backslash U_1$, a $U_2 = U_1 \cup \ball{x_2}{\eps}$. Ogólnie $x_n \in K \backslash U_{n-1}$ i$U_n = U_{n-1} \cup \ball{x_{n}}{\eps}$. Oczywiście $K \backslash U_{n-1} \neq \emptyset$, gdyż gdyby było inaczej, to z konstrukcji mielibyśmy wbrew założeniu skończoną $\eps$ - sieć pokrywającą $K$. \\
         Otrzymaliśmy ciąg $\ciag{x}$, który jednak nie może mieć podciągu zbieżnego, gdyż nie spełnia warunku Cauchy'ego - z konstrukcji jasno widać, że $\forall_{i < j}: d(x_i, x_j) \geqslant \eps$, jako że $x_j \in K \backslash U_{j-1} \subseteq K \backslash U_i$. Przeczy to założeniu, że $K$ jest ciągowo zwarty.
    \end{dow}


    \begin{twier}{} \label{twier:przel-gest}
        Jeśli $K$ jest ciągowo zwarty, to istnieje przeliczalny $D \subseteq K$, t.ż. $\overline{D} = K$ i.e. $D$ jest gęsty w $K$.
    \end{twier}

    \begin{dow}
        Weźmy \begin{equation*}
            D = \bigcup_{n = 1}^{\oo} S_{\frac{1}{n}} \subseteq K,
        \end{equation*} gdzie jako $S_\eps$ oznaczyliśmy skończoną $\eps$ - sieć w $K$, która na mocy Tw. \ref{twier:sk-eps-siec} istnieje. Jasno widać, że $D$ jest przeliczalny. Wtedy $\overline{D} \subseteq K$, gdyż dla każdego $y \in K$ tworzymy ciąg $\ciag{x}$ elementów, takich że $x_n \in S_{\frac{1}{n}}$ i $y \in \ball{x_n}{\frac{1}{n}}$, co zawsze możemy zrobić, bo $S_\frac{1}{n}$ są $\eps$ - sieciami w $K$. Wtedy $d(x_n, y) < \frac{1}{n} \to 0$, więc $x_n \to y$, czyli $y 
        \in \overline{D}$. Odwrotnie mamy $\overline{D} \supseteq K$, bo $D \subseteq K \implies \overline{D} \subseteq \overline{K}$ (domknięcie jest monotoniczne, a $K$ jest domknięty por. Tw. \ref{twier:ciag-zwarty-domk}). Z tych dwu inkluzji mamy tezę. 
    \end{dow}



    \begin{twier}{} \label{twier:przel_podpokr}
        Jeśl $K$ jest ciągowo zwarty, do każde jego pokrycie otwarte ma przeliczalne podpokrycie. 
    \end{twier}

    \begin{dow}
        Niech $U$ - pokrycie otwarte $K$. Dla dowolnego $y \in K$ zdefiniujmy:\begin{equation*}
            n_y = \min\{ n \in \N \, | \, \exists_{U_y \in U}: \ball{y}{\frac{1}{n}} \subseteq U_y \}.
        \end{equation*}  Taka liczba będzie bez wątpienia istniała, gdyż $U$ pokrywa $K$ i składa się z samych zbiorów otwartych. Niech $U_y$ będzie zbiorem spełniającym powyższy warunek dla $y \in K$. Niech $V = \{ U_y \, | \, y \in D \}$, gdzie $D$ to przeliczalny podzbiór gęsty w $K$ (istnieje on na mocy Tw. \ref{twier:przel-gest}). Jasne, że $V \subseteq U$ i że $V$ jest przeliczalny. Pokażemy, że jest to podpokrycie. \\ 
        Niech $x \in K$. Wiemy, że $\exists_{U_x \in U, \, N \in \N}: \ball{x}{\frac{1}{N}} \subseteq U_x$. Ponadto, jako że $D$ jest gęsty w $K$, to $\exists_{y \in D}: d(y, x) < \frac{1}{2N}$. Wtedy: \begin{equation*}
            x \in \ball{y}{\frac{1}{2N}} \implies \ball{y}{\frac{1}{2N}} \subseteq \ball{x}{\frac{1}{N}} \subseteq U_x \in U.
        \end{equation*} Więc $2N \geqslant n_y$ z konstrukcji $n_y$. Zatem: \begin{equation*}
            x \in \ball{y}{\frac{1}{2N}} \subseteq \ball{y}{\frac{1}{n_y}} \subseteq U_y \in V.
        \end{equation*} Czyli $V$ pokrywa $K$.
    \end{dow}

    Jesteśmy już gotowi, żeby udowodnić najważniejsze twierdzenie w tej sekcji.

    \begin{twier}{Zwartość ciągowa = zwartość pokryciowa}
        Zbiór $K$ jest ciągowo zwarty wtedy i tylko wtedy gdy jest zwarty (pokryciowo).
    \end{twier}






    \begin{dow}
        \paragraph{$\implies$} Niech $K \subseteq \subseteq X$, $\ciag{x}$ będzie ciągiem elementów z $K$ i niech $Z  = \{ x_n \, | \, n \in \N \}$. Pokażemy, że zachodzi jedna z dwu możliwości: \begin{enumerate}
            \item \begin{equation*}
                \exists_{y \in K}: \forall_{U \in \Nau(y), \, U \subotw X}: | Z \cap U | = \oo. 
            \end{equation*}
            \item $Z$ jest skończony.
        \end{enumerate}  
        Mianowicie, niech $\neg 1.$, i.e.: \begin{equation*}
            \forall_{y \in K}: \exists_{U_y \in \Nau(y), \, U_y \, \subotw X}: | Z \cap U_y | < \oo.
        \end{equation*} Oczywiście $\{ U_y \}_{y \in K}$ jest pokryciem otwartym $K$ (gdyż każdy $y$ zawiera się np. w odpowiadającym mu $U_y$). Wybierzmy więc zeń skończone podpokrycie, tak, żeby: \begin{equation*}
            K \subseteq U_{y_1} \cup U_{y_2} \cup ... \cup U_{y_N}. 
        \end{equation*} Mamy jednak $Z \subseteq K$, zatem: \begin{equation}
            Z = K \cap Z = (U_{y_1} \cap Z) \cup (U_{y_2} \cap Z) \cup ... \cup (U_{y_N} \cap Z).
        \end{equation} Każdy z elementów tej sumy jest skończony, na mocy założenia, zatem $Z$ także jest skończony, jako skończona suma skończonych. \\
        Widzimy więc, że zachodzi 1. lub 2.: \begin{enumerate}
            \item Jeśli zachodzi pierwszy warunek, to skoro: \begin{equation*}
                \exists_{y \in K}: \forall_{U \in \Nau(y), \, U \subotw X}: | Z \cap U | = \oo,
            \end{equation*} to konstruujemy $k \mapsto n_k$, tak aby $\forall_k: n_{k+1} > n_k$ i $x_{n_k} \in \ball{y}{\frac{1}{k}}$. Jest to możliwe, bo każde otwarte otoczenie $y$ zawiera nieskończenie wiele elementów ciągu $x_n$, w szczególności zawiera element o indeksie większym od dowolnej liczby. Oczywiście z podanej konstrukcji mamy $x_{n_k} \to y \in K$.
            \item Jeśli $Z$ jest skończony, to z zasady szuflatkowej istnieje wartość $x \in Z \subseteq K$, która zostanie odwiedzona nieskończenie wiele razy przez wyrazy ciągu $x_n$ - można więc wziąć podciąg stały równy $x$ i oczywiście do tej liczby zbieżny.
        \end{enumerate} Widzimy więc, że w obu przypadkach potrafimy skonstruować podciąg zbieżny w $K$, zatem $K$ istotnie jest ciągowo zwarty.
        
        \paragraph{$\impliedby$} Dowód przeprowadzimy nie wprost. Niech $K$ będzie ciągowo zwartym, $U$ pewnym jego pokryciem otwartym, a $V = {V_i}_{i \in \N}$ jego przeliczalnym podpokryciem, które na mocy Tw. \ref{twier:przel_podpokr} istnieje. Przypuśćmy, że żadna skończona podrodzina $V$ nie jest pokryciem $K$. Wtedy dla $n \in \N$ wybieramy $x_n \in K \backslash (\bigcup_{i=1}^{n}V_i)$. Oczywiście zbiór z którego wybieramy $x_n$ będzie niepusty na mocy założenia, że żadna skończona podrodzina $V$ nie pokrywa $K$. Otrzymujemy ciąg $\ciag{x}$ elementów z $K$, więc ma on podciąg zbieżny do $y \in K$. Ale $\exists_{N}: y \in V_N$. Skoro jednak $V_N$ jest otwarty (gdyż wyjściowe pokrycie $U$ było otwarte), to prawie wszystkie wyrazy podciągu $\ciag{x}$ zbieżnego do $y$ powinny leżeć w $V_N$. Jednakże tylko skończona liczba wyrazów $x_n$ leży w $V_N$, gdyż dla $n \geqslant N$: $x_n \in K \backslash V_N$. Sprzeczność! Musi więc istnieć skończone podpokrycie $V$, zatem $K$ jest zwarty pokryciowo.  


    \end{dow}

    \begin{obs}{} \label{obs:domk-podzb-zwar}
        Domknięty podzbiór zbioru zwartego jest zwarty.
    \end{obs}
    %% TODO proper dowód
    \paragraph*{} Istotnie $K \subseteq \subseteq X$ $\wedge$ $C \subseteq K$ $\wedge$ $C = \ol{C}$ $\implies$ $C \subseteq \subseteq X$.

    \begin{twier}{Bolzano-Weierestrassa II} \label{twier:BL II}
        Każdy ograniczony i domknięty podzbiór $\R$ jest zwarty.
    \end{twier}

    \begin{dow}
        Wynika on prosto z równoważności zwartości ciągowej i pokryciowej. Jeśli jakiś podzbiór $\R$ jest ograniczony, to zawiera podciąg zbieżny (Twierdzenie \ref{twier:BL I}), a skoro jest domknięty, to ów podciąg zbieżny ma granicę w tymże zbiorze. Zatem podzbiór ten jest ciągowo zwarty, czyli zwarty.
    \end{dow}


    \begin{defr}{Metryka na iloczynie kartezjańskim}
        Rozważmy dwie p-ń metryczne $(X, d_1)$ i $(Y, d_2)$. Można wtedy wprowadzić na ich iloczynie kartezjańskim nową metrykę $D$, tworząc p-ń. $(X \times Y, D)$. Standardowo robi się to w ramach jednej z równoważnych metryk $d_p$ ($p \in [1, \oo]$) - tak, że $D((x_1, y_1), (x_2, y_2)) = d_p(d_1(x_1, x_2), d_2(y_1, y_2))$. Wszystkie te metryki zadają tę samą topologię. Metrykę na iloczynie kartezjańskim większej liczby zbiorów definiuje się indukcyjnie.
    \end{defr}
    \begin{obs}{}
        \begin{equation}
            (x_n, y_n) \xrightarrow[]{D} (x, y) \iff x_n \to x \wedge y_n \to y.
        \end{equation} Widzimy, że zbieżność na iloczynie kartezjańskim jest "po współrzędnych".
    \end{obs}

    \begin{twier}{} \label{twier:il-kart-zwar}
        Iloczyn kartezjańskich dwu zbiorów zwartych jest zwarty.
    \end{twier}

    \begin{dow}
        %TODO
        TODO
    \end{dow}


    \begin{obs}{} \label{obs:zwar-kres}
        Niech $K \subseteq \subseteq \R$. Wtedy $K$ zawiera swoje kresy. 
    \end{obs}
    \paragraph{} Rzeczywiście, skoro $K$ jest zwarty, to i ograniczony: $\inf K \in \R$ i $\sup K \in \R$. Ponadto $\inf K$ i $\sup K$ są granicami ciągów z $K$, więc do $K$ należą, skoro $K$ jest domknięty.

    \begin{twier}{}
        Niech $(X, d)$ to p-ń. metryczna zupełna. Niech $K \subdomk X$. Wtedy, zbiór $K$ jest zwarty jeżeli $\forall_{\eps >0}$ można go pokryć skończoną $\eps$ - siecią.
    \end{twier}
%TODO
    \begin{dow}
        TODO
    \end{dow}


\begin{twier}[label=twier:ob-zwar-zwar]{Obraz zbioru zwartego jest zwarty}
    Niech $(X, d)$, $(Y, \delta)$ będą p-ń. metrycznymi, a $\varphi: X \to Y$ będzie odwzorowaniem ciągłym. Ponadto niech $K \subseteq \subseteq X$. Wtedy zbiór $\varphi(K)$ jest zwarty.
\end{twier}


\begin{dow}
    Niech $U = \{ U_i \}_{i \in I}$ - rodzina zbiorów otwartych pokrywająca $\varphi(K)$: $\varphi(K) \se \bigcup_{i \in I} U_i $. Skonstruujmy rodzinę $V = \{ V_i = \varphi^{-1}(U_i) \}$. Skoro $\varphi$ jest ciągłe, to $V$ jest rodziną otwartą. Podnadto z włąsciwości przeciwobrazów: $K \se \bigcup_{i \in I}V_i$. Zatem $V$ jest pokryciem otwartym. Możemy więc zeń wybraż skończone podpokrycie otwarte: \begin{equation*}
        K \se V_{i_1} \cup .. \cup V_{i_n} = \varphi^{-1}(U_{i_1}) \cup .. \cup \varphi^{-1}(U_{i_n}).
    \end{equation*} Wtedy: \begin{equation*}
        \varphi(K) \se \varphi(\varphi^{-1}(U_{i_1}) \cup .. \cup \varphi^{-1}(U_{i_n})) = \varphi(\varphi^{-1}(U_{i_1})) \cup .. \cup \varphi(\varphi^{-1}(U_{i_n})) \se U_{i_1} \cup ... \cup U_{i_n}.
    \end{equation*} Zatem dla każdego pokrycia $\varphi(K)$ wybraliśmy zeń skończone podpokrycie - zatem $\varphi(K)$ jest zwarty.
\end{dow}

\begin{obs}{Odwzorowanie ciągłe na zbiorze zwartym osiąga swoje kresy}
    Niech $(X, d)$ - p-ń. metryczna i niech $f: X \to \R$ będzie ciągłą funkcją oraz niech $K \se \se X$. Wtedy: \begin{equation}
        \sup_{x \in K} f = \sup f(K) \in f(K) 
    \end{equation} oraz\begin{equation}
        \inf_{x \in K} f = \inf f(K) \in f(K).
    \end{equation}
\end{obs}
    Wynika to od razu z Twierdzenia \ref{twier:ob-zwar-zwar} oraz Obserwacji \ref{obs:zwar-kres}.
    
    
\begin{twier}{Heinego - Borela}
    Podzbiór $K \se \R ^n $ jest zwarty wtedy i tylko wtedy, gdy jest domknięty i ograniczony.
\end{twier}
\begin{dow}
    \paragraph{$\implies$} Por. Tw. \ref{twier:zwar-domk}.
    \paragraph{$\impliedby$} Skoro $K$ jest ograniczony, to istnieje kostka postaci:\begin{equation}
        C = [a_1, b_1] \times ... \times [a_n, b_n],
    \end{equation} t.ż $K \se C$. Ponadto, $C$ musi być zwarta, jako iloczyn kartezjański dwu zbiorów zwartych (por. Tw. \ref{twier:il-kart-zwar}). Zauważyliśmy jednak, żę domknięty podzbiór zbioru zwartego jest zwarty, zatem i $K$ jest zwarty (Obs. \ref{obs:domk-podzb-zwar}).
\end{dow}

\begin{twier}{Lemat Lebesgue'a}
    Niech $(X, d)$ - p-ń. metryczna, $K \se \se X$ ($K$ jest zwartym podzbiorem $X$). Niech $U$ będzie pokryciem otwartym $K$. Wówczas: \begin{equation}
        \exists_{\lambda > 0} \forall_{x \in K} \exists_{U_x \in U}: \ball{x}{\lambda} \subseteq U_x.
    \end{equation}
\end{twier}
    \paragraph{Uwaga!} Liczbę $\lambda$ podaną w twierdzeniu nazywa się \textbf{liczbą Lebesgue'a}.

\begin{dow}
    Skoro $K$ - zwarty, to wybierzmy z $U$ skończone podpokrycie: $\{U_i\}_{1 \leq i \leq N}$. 
    \begin{itemize}
        \item Jeżeli $\exists_{i\leq N}: X = U_i$, to sytuacja nie przedstawia żadnego problemu - za $\lambda$ bierzemy dowolną liczbę, zaś za $U_x = X$. Teza jest trywialnie spełniona.

        \item Jeżeli $\forall_{i \leq N}: X \not \se U_i$, to definiujemy rodzinę $F_i = X \backslash U_i$ oraz funkcję: \begin{equation*}
            f(x) = \frac{1}{N} \bsum{i = 1}{N}d(x, F_i),
        \end{equation*} gdzie \begin{equation*}
            d(x, F_i) = \inf_{v \in F_i}\{d(x, v)\}
        \end{equation*} jest dobrze określoną funkcją dla niepustego $F_i$.\\
        Oczywiście $f$ jest ciągła, bo mamy: \begin{equation*}
            \big| f(x) - f(x_0)\big| = \frac{1}{N} \big|\bsum{i = 1}{N} d(x, F_i) - d(x_0, F_i) \big| \leq \frac{1}{N} \bsum{i = 1}{N} \big| \inf_{v \in F_i}\{d(x, v)\} - \inf_{v \in F_i}\{d(x_0, v) \}\big|.
        \end{equation*} Zauważmy, że skoro: \begin{equation*}
            d(x, v)  \leq d(x_0, v) + d(x, x_0) 
        \end{equation*} oraz: \begin{equation*}
            d(x_0, v) \leq d(x, v) + d(x, x_0),
        \end{equation*} to: \begin{equation*}
            \inf_{v \in F_i}\{d(x, v)\} \leq \inf_{v \in F_i}\{d(x_0, v) \} + d(x, x_0)
        \end{equation*} i: \begin{equation*}
            \inf_{v \in F_i}\{d(x_0, v)\} \leq \inf_{v \in F_i}\{d(x, v) \} + d(x, x_0),
        \end{equation*} zatem: \begin{equation*}
            |\inf_{v \in F_i}\{d(x, v)\} - \inf_{v \in F_i}\{d(x_0, v) \}| \leq d(x, x_0).
        \end{equation*} Widzimy, więc że \begin{equation*}
            |f(x) - f(x_0)| \leq d(x, x_0),
        \end{equation*} czyli $f$ musi być ciągła. \\
        Skoro $f$ jest ciągła na zbiorze zwartym $K$ i $\forall_{x \in K}: f(x) > 0$ (bo inaczej -- gdyby $f(x) = 0$, to $x$ nie nalezałby do żadego pokrycia $U_i$), to $\exists_{x_0}: f(x_0) = \inf f(x) >0$. Funkcja ciągła na zbiorze zwartym osiąga swoje kresy. Weźmy $\lambda = \inf f(x) > 0$. Skoro mamy $\forall_{x \in K}: f(x) \geq \lambda$, a $f(x)$ jest średnią odległością $x$ od $F_i$, to w szczególności istnieje takie $F_j$, że $\lambda \leq d(x, F_j) = \mathlarger{\inf}_{v \not \in U_j}d(x, v)  \implies \forall_{v \not \in U_j}: d(x, v) \geq \lambda$.  Zatem dla tego $U_j$:\begin{equation*}
            \ball{x}{\lambda} \se U_j.
        \end{equation*} Znaleźliśmy więc szukaną $\lambda$ i szukane $U_x = U_j$. Dowód został zakończony.
    \end{itemize}
\end{dow}


\subsection{Spójność}

\begin{defr}{Spójność}
    Powiemy, że zbiór $Z$ w pewnej metryce jest \textbf{niespójny}, jeśli:\begin{equation}
        \exists_{Z_1, Z_2 \neq \emptyset}: Z_1 \cup Z_2 = Z \quad \wedge \quad  Z_1 \cap \ol{Z_2} = \emptyset.
    \end{equation} W przeciwnym razie, zbiór nazwiemy \textbf{spójnym}. Dla zbiorów spójnych, mamy: \begin{equation*}
        Z_1, Z_2 \neq \emptyset \wedge Z_1 \cup Z_2 = Z \implies Z_1 \cap \ol{Z_2} \neq \emptyset \vee \ol{Z_1} \cap Z_2.
    \end{equation*}
\end{defr}

\begin{obs}{}
    Podzbiór $\R$ jest spójny wtedy i tylko wtedy, gdy jest przedziałem.
\end{obs}


%TODO dowód

\begin{twier}{Obraz zbioru spójnego jest spójny}
    Jeśli $Z$ jest zbiorem spójnym, a $\varphi: X \to Y$ odzworowaniem ciągłym, to $\varphi(Z)$ jest spójny.
\end{twier}

\begin{dow}
    Dowód nie wprost. Załóżmy, że $\varphi(Z)$ jest niespójny, tj. istnieją takie niepuste $W_1, W_2 \se Y$, że $W_1 \cap \ol{W_2} = \emptyset$ i $W_1 \cup W_2 = \varphi(Z)$. Niech $Z_1 = \varphi^{-1}(W_1) \cap Z$ i $Z_2 = \varphi^{-1}(W_1) \cap Z$. Mamy: \begin{equation*}
        Z \se \varphi^{-1}(\varphi(Z)) = \varphi(W_1) \cup \varphi(W_2).
    \end{equation*} Zatem $Z = Z_1 \cup Z_2$. Skoro $Z$ jest spójny, to $\ol{Z_1} \cap Z_2 \neq \emptyset$ lub $Z_1 \cap \ol{Z_2} \neq \emptyset$. Bez straty ogólności, załóżmy ten pierwszy przypadek. Wtedy istnieje $\ciag{x}$, t.ż. $x_n \in Z_1$ i $x_n \to x \in Z_2$. Ale wtedy -- z ciągłości $\varphi$ -- $W_1 \ni \varphi(x_n) \to \varphi(x) \in W_2$, zatem $\ol{W_1} \cap W_2 \neq \emptyset$, wbrew założeniu. Powstała sprzeczność dowodzi tezy.
\end{dow}

\begin{obs}{Łukowa spójność}
    Zbiór spójny łukowo jest spójny.
\end{obs}


%TODO wiecej na ten temat



\newpage
\section{Rachunek różniczkowy}

    \begin{defr}{Różniczkowalność}
        Niech $f: I \to \R$, gdzie $I$ jest przedziałem w $\R$ (otwartym lub domkniętym). Powiemy, że $f$ jest różniczkowalna w punkcie $x_0 \in I$, jeśli istnieje granica: \begin{equation}
            \lim_{I \ni x \to x_0} \frac{f(x) - f(x_0)}{x - x_0}.
        \end{equation} W szczególności, jeśli $I = [a, b]$, to $f$ jest różniczkowalna w $a$, jeśli istnieje granica: \begin{equation}
            \lim_{I \ni x \to a^+} \frac{f(x) - f(a)}{x - a}.
        \end{equation} Wartość tychże granic nazywamy pochodną funkcji $f$ w punkcie $x_0$ i oznaczamy $f'(x_0)$. 
    \end{defr}

    \begin{defr}{Pochodna}
        Jeśli w każdym punkcie funkcji $f$ istnieje skończona pochodna, to funkcje $f': I \to \R$, $f': x \mapsto f'(x)$ nazywamy \textbf{pochodną} funkcji $f$. Mówimy wtedy o $f$, że jest \textbf{różniczkowalna}.
    \end{defr}

    \begin{twier}[label=twier:poch-do-0]{}
        Niech $x \in I \deg$. Wtedy:\begin{equation}
            \exists_{f'(x_0)} \iff \exists_{y}: \lim_{h \to 0} \frac{f(x_0+h) - f(x_0) - y h}{h} = 0.
        \end{equation} Wtedy oczywiście $y = f'(x)$.
    \end{twier}

    \begin{dow}
        \paragraph{$\implies$} Jeśli $\lim_{x \to x_0} \frac{f(x) - f(x_0)}{x - x_0} = f'(x_0)$, to $\lim_{h \to 0} \frac{f(x_0 + h) - f(x_0)}{h} = f'(x_0)$ i: \begin{equation*}
            \lim_{h \to 0} \frac{f(x_0 + h) - f(x_0) - h f'(x_0)}{h} = 0.            
        \end{equation*}

        \paragraph{$\impliedby$} Jeśli: \begin{equation*}
            \lim_{h \to 0} \frac{f(x_0 + h) - f(x_0) - h  y}{h} = 0,
        \end{equation*} to: \begin{equation*}
            \lim_{x \to x_0} \frac{f(x) - f(x_0)}{h} = y = f'(x_0).
        \end{equation*}
    \end{dow}

    \begin{twier}{Funkcja różniczkowalna jest ciągła}
        Jeśli funkcja $f$ jest różniczkowalna w pewnym punkcie $x_0$, to jest w tym punkcie ciągła.
    \end{twier}

    \begin{dow}
        Skoro $\lim_{x \to x_0} \frac{f(x) - f(x_0)}{x - x_0} = f'(x_0)$, to $\lim_{x \to x_0} f(x) - f(x_0)= lim_{x \to x_0} f'(x_0) (x - x_0) 0$, zatem funkcja jest ciągła.

    \end{dow}

    \begin{obs}{Podstawowe własności pochodnych}
        Niech $f, g: I \to \R$ będą funkcjami różniczkowalnymi. Wtedy:\begin{itemize}
            \item $(f + g)'$ istnieje i: \begin{equation}
                (f + g)' = f' + g'.
            \end{equation}
            \item Jeśli $a \in R$, to $(a f)'$ istnieje i: \begin{equation}
                (a f)' = a f'.
            \end{equation}
            \item $(fg)'$ istnieje i: \begin{equation}
                (fg)' = f' \cdot g + f \cdot g'.
            \end{equation} Jest to tzw. wzór Leibnitza.
            \item Jeśli $g(x) \neq 0$ i $g'(x) \neq 0$ dla wszyskich $x \in I$, to $(\frac{f}{g})'$ istnieje i: \begin{equation}
                (\frac{f}{g})' = \frac{f' g - f g'}{(g)^2}.
            \end{equation}
        \end{itemize}
    \end{obs}

    %TODO dowód w wolnym czasie (chociaż jest oczywisty)

    \paragraph{} Dowód jest bardzo prosty i pozostawiamy go jako ćwiczenie.
    Nieco ciekawsze (i dużo ważniejsze) jest następujące: 
    \begin{twier}{Pochodna funkcji złożonej}
        Niech $g: I_1 \to I_2$, $f: I_2 \to \R$ będą funkcjami różniczkowalnymi odpowiednio w punktach $x_0 \in I_1 \deg$ i $g(x_0) \in I_2 \deg$. Wtedy $f \circ g$ także jest różniczkowalna i w $x_0$: \begin{equation}
            (f \circ g)' (x_0) = f'(g(x_0)) \cdot g'(x_0).
        \end{equation} Reguła ta nazywa się czasem \textbf{regułą łańuchową}.
    \end{twier}

    \begin{dow}
        Niech  $r(h) = g(x_0 + h) - g(x_0) - g'(x_0)h$. Wtedy (por. Tw. \ref{twier:poch-do-0}): \begin{equation}
            \lim_{h \to 0} \frac{r(h)}{h} = 0.
        \end{equation} Niech $y_0 = g(x_0)$ i $R(k)= f(y_0 + h) - f(y_0) - f'(y_0) h.$ Znowuż: \begin{equation}
            \lim_{k \to 0}\frac{R(k)}{k} =0.
        \end{equation} Wtedy: \begin{equation}
            f(g(x_0 + h)) - f(g(x_0)) = f(g(x_0) + g'(x_0) h + r(h)) - f(g(x_0)) = f(g(x_0) + k(h)) - f(g(x_0)),
        \end{equation} gdzie oznaczyliśmy $k(h) =  g'(x_0) h + r(h)$. Oczywiście $\lim_{h \to 0} k(h) = 0$. Licząc dalej: \begin{equation}
            f(g(x_0 + h)) - f(g(x_0)) =  f(g(x_0) + k(h)) - f(g(x_0)) = R(k(h)) + f'(y_0)k(h).
        \end{equation} Zatem: \begin{equation*}
            \lim_{h\ to 0} \frac{f(g(x_0 + h)) - f(g(x_0))}{h} = f'(y_0) \lim_{h \to 0} \frac{k(h)}{h} + \lim_{h \to 0} \frac{R(k(h))}{h}.
        \end{equation*} Zauważmy, że:\begin{equation*}
            \lim_{h \to 0} \frac{k(h)}{h} = \lim_{h \to 0} \frac{g'(x_0) h + r(h)}{h} = g'(x_0).
        \end{equation*} Stąd mamy: \begin{equation*}
            \lim_{h \to 0} \frac{R(k(h))}{h} = \lim_{h \to 0} \frac{R(k(h))}{k(h)} \frac{k(h)}{h} = 0.
        \end{equation*} Ostatecznie więc: \begin{equation*}
            \lim_{h\ to 0} \frac{f(g(x_0 + h)) - f(g(x_0))}{h} =  f'(y_0) g'(x_0).
        \end{equation*} 
    \end{dow}


    \begin{defr}{Inna definicja pochodnej}
        Niech $f: \R^n \to \R^m$. Niech $\vec{x_0} \in \R^n$. Jeśli istnieje taka macierz $A \in M(\R)_{m \times n}$, że: \begin{equation}
            \lim_{|| \vec{h} || \to 0} \frac{f(\vec{x_0} + \vec{h}) - f(\vec{x_0}) - A \vec{h}}{|| \vec{h} ||} = 0,
        \end{equation} dla $ \vec{h} \in \R^n$, to macierz $A$ nazwiemy pochodną $f$ w punkcie $\vec{x_0}$. Łatwo widać, że $A$ jest wyznaczone jednoznacznie.
    \end{defr}


    \paragraph{} Zajmiemy się teraz związkiem pochodnej z własnościami funkcji rzeczywistych.

    \begin{defr}{Ekstrema}
        Niech $(X, d)$ - p-ń. metryczna oraz $f: X \to \R$. Wtedy $x_0 \in X$ nazwiemy: \begin{itemize}
            \item \textbf{Lokalnym minimum}, jeśli: \begin{equation}
                \exists_{U \in \Nau(x_0)} \forall_{x \in U}: f(x) \geq f(x_0).
            \end{equation}

            \item \textbf{Lokalnym minimum ścisłym}, jeśli: \begin{equation}
                \exists_{U \in \Nau(x_0)} \forall_{x \in U}: f(x) > f(x_0).
            \end{equation}

            \item \textbf{Lokalnym maksimum}, jeśli: \begin{equation}
                \exists_{U \in \Nau(x_0)} \forall_{x \in U}: f(x) \leq f(x_0).
            \end{equation}

            \item \textbf{Lokalnym maksimum ścisłym}, jeśli: \begin{equation}
                \exists_{U \in \Nau(x_0)} \forall_{x \in U}: f(x) < f(x_0).
            \end{equation}

        \end{itemize} Wpólna nazwa, na którąś z tych sytuacji, to wystąpienie lokalnego \textbf{ekstremum} (ściłego ekstremum). 
    \end{defr}

    \begin{twier}{Warunek konieczny istnienia ekstremum}
        Jeśli $f: [a, b] \to \R$ jest w punkcie $x_0$ różniczkowalna i $x_0$ jest ekstremum $f$, to: \begin{equation}
            f'(x_0) = 0.
        \end{equation}
    \end{twier}

    \begin{dow}
        Założmy, że $f$ ma w $x_0$ minimum. Wtedy oczywiście: \begin{equation*}
            f(x) - f(x_0) \geq 0. 
        \end{equation*} Zatem: \begin{equation*}
            \frac{f(x) - f(x_0)}{x - x_0} \geq 0,
        \end{equation*} dla $x > x_0$, więc: \begin{equation*}
            \frac{f(x) - f(x_0)}{x - x_0} \xrightarrow{x \to x_0^+} f'(x_0) \geq 0.
        \end{equation*} Jednakże: \begin{equation*}
            \frac{f(x) - f(x_0)}{x - x_0} \leq 0,
        \end{equation*} dla $x < x_0$ i:\begin{equation*}
            \frac{f(x) - f(x_0)}{x - x_0} \xrightarrow{x \to x_0^-} f'(x_0) \leq 0.
        \end{equation*} Jedyną możliwością połączenia tych nierówności jest $f'(x_0) = 0$. Dla maksimum w $x_0$ dowód zupełnie analogiczny.
    \end{dow}


    Twierdzenie to pomoże nam udowodnić kilka zasadniczych twierdzeń związanych z pochodnymi. Zaczniemy od:
    \begin{twier}{Rolle'a}
        Jeśli $f: [a, b] \to \R$ jest ciągła i różniczkowalna na $]a, b[$ oraz $f(a) = f(b)$, to: \begin{equation*}
            \exists_{\xi \in ]a, b[ }: f'(\xi) = 0. 
        \end{equation*}
    \end{twier}

    \begin{dow}
        $f$ jest ciągła na zwartej dziedzinie $[a, b]$, także osiąga swoje kresy. Jeśli $\sup f = \inf f = f(a) = f(b)$, to $f$ jest funkcją stałą: $f := f(a)$ i jej pochodna we wszystkich punktach pomiędzy $a$ i $b$ jest równa 0. W przeciwnym wypadku, (i.e. $\sup f \neq f(a)$ lub $\inf f \neq f(a)$) istnieje $\xi \in ]a, b[$ osiągające ów kres różny od wartości w $a$. Oczywiście $\xi$ będzie ektremum, zatem z poprzedniego twierdzenia $f'(\xi) = 0$. 
    \end{dow}

    Uogólnieniami tych twierdzeń są następujące dwa rezultaty:

    \begin{twier}{Cauchy'ego}
        Niech $f, g: [a, b] \to \R$ - ciągłe i różniczkowalne na $]a, b[$, wtedy:\begin{equation}
            \exists_{\xi \in ]a, b[ }: g'(\xi) (f(b) - f(a)) = f'(\xi) (g(b) - g(a)).
        \end{equation} 
    \end{twier}


    \begin{dow}
        Konstruujemy funkcję: \begin{equation*}
            h(x) = g(x) (f(b) - f(a)) - f(x) (g(b) - g(a)).
        \end{equation*} Łatwo widzieć, że: \begin{equation*}
            h(a) = h(b) = g(a)f(b) - f(b) g(a).
        \end{equation*} Ponadto $h$ powstaje z operacji arytmetycznych na $f, g$ więc również jest ciągła i różniczkowalna na $]a, b[$. Spełnia więc wszystkie założenia Tw. Rolle'a: \begin{equation*}
            \exists_{\xi \in ]a, b[ }: h'(\xi) = 0
        \end{equation*} Ale: \begin{equation*}
            h'(\xi) = g'(\xi) (f(b) - f(a)) - f'(\xi) (g(b) - g(a)).
        \end{equation*} Stąd prosto otrzymujemy tezę.
    \end{dow}

    \begin{twier}{Lagrange'a}
        Niech $f: [a, b] \to \R$ - ciągła i różniczkowalne na $]a, b[$, wtedy:\begin{equation}
            \exists_{\xi \in ]a, b[ }:f'(\xi)  =  \frac{f(b) - f(a)}{b - a}.
        \end{equation} 
    \end{twier}

    \begin{dow}
        Bierzemy $g = x$ w Tw. Cauchy'ego. Oczywiście $g$ spełnia wszystkie założenia. Stąd: \begin{equation}
            \exists_{\xi \in ]a, b[ }:  f(b) - f(a) = f'(\xi) (b - a).
        \end{equation} Wystarczy podzielić teraz przez $(b - a)$.
    \end{dow}

    %TODO reszta rachunku różniczkowego


\newpage
\section{Rachunek całkowy}
\subsection{Całka Riemanna}
W toku tego podrozdziału zajmować się będziemy jedynie funkcjami ograniczonymi na zwartych przedziałach - gdyż dla takich właśnie definiuje się całkę Riemanna.
\begin{defr}{Podział przedziału}
    \textbf{Podziałem} przedziału $[a, b]$  o długości $n$ nazywamy ciąg skończony $\pi = (t_0, ..., t_n)$, t.ż. $a = t_0 < t_1 < ... < t_n = b$. Powiemy też, że $\pi' = (t_0, ..., t_n)$ jest drobniejszy niż $\pi = (s_0, ..., s_m)$ ($\pi' \preceq \pi$ ) jeśli $\{t_0, .., t_n\} \se \{ s_0, ..., s_m\}$.
\end{defr} 
\begin{defr}{Suma górna i dolna}
    Niech $f:[a, b] \to \R$ będzie funkcją ograniczoną. Wtedy definiujemy dla niej i pewnego podziału $\pi$ przedziału $[a, b]$: \begin{itemize}
        \item \textbf{Sumę dolną}: \begin{equation}
            \Sd (f, \pi) = \sum_{i = 1}^{n} (t_i - t_{i - 1})\inf_{\przedz{t}{i}} f  
        \end{equation} oraz \tb{Sumę górną}: \begin{equation}
            \Sg (f, \pi) = \sum_{i = 1}^{n} (t_i - t_{i - 1})\sup_{\przedz{t}{i}} f  
        \end{equation}
    \end{itemize}
\end{defr}

\begin{twier}{}
    Ustalmy przedział $[a, b]$.
    \begin{enumerate}
        \item Jeśli $\pi_1 \preceq \pi_2$ to: \begin{equation}
            \Sg (f, \pi_1) \leq \Sg(f, \pi_2) 
        \end{equation} oraz: \begin{equation}
            \Sd (f, \pi_1) \geq \Sd(f, \pi_2).
        \end{equation}
        \item Dla każdych dwu podziałów $\pi_1, \pi_2$ istnieje podział $\pi'$ t.ż. $\pi' \preceq \pi_1, \pi_2$.
        \item Dla każdych dwu podziałów $\pi_1, \pi_2$: \begin{equation}
            (b - a) \inf_{[a, b]} f \leq \Sd(f, \pi_1) \leq \Sg(f, \pi_2) \leq (b - a) \sup){[b, a]} f.
        \end{equation}
    \end{enumerate}
\end{twier}

\begin{dow}
    \begin{itemize}
        \item Jest to prosta konsekwencja faktu, że jeśli $c \in [a, b]$ to:\begin{equation*}
            (c - a) \sup_{[a, c]} f + (b -c) \sup_{[c, b]} f \leq (b - a) \sup_{[a, b]} f 
        \end{equation*} i analogicznej własności dla inf.
        \item Za $\pi'$ wystarczy wziąć sumę teoriomnogościową punktów z $\pi_1$ i $\pi_2$.
        \item Stosujemy punkt 1. do $\pi'$ drobniejszego od obu podziałów.
    \end{itemize}
\end{dow}


\begin{defr}{Całka dolna i górna}
    \tb{Całką dolną} funkcji ograniczonej $f$ na przedziale $[a, b]$ nazywamy liczbę:\begin{equation}
        \Cd_{a}^{b} f = \sup_{\pi \preceq [a, b]} \Sd(f, \pi), 
    \end{equation} to jest supremum sum dolnych po wszyskich podziałach $[a, b]$. Analogicznie definiujemy całkę górną: \begin{equation}
        \Cg_{a}^{b} f = \inf_{\pi \preceq [a, b]} \Sd(f, \pi).
    \end{equation}
\end{defr}

\begin{defr}{Całkowalność w sensie Riemanna}
    Powiemy, że funkcja ograniczona $f$ na przedziale $[a, b]$ jest \tb{całkowalna w sensie Riemanna}, jeśli jej całka dolna jest równa całce górnej. Ich wspólną wartość nazwiemy po prostu całką z $f$ po przedziale $[a, b]$: \begin{equation}
        \int_{a}^{b} f  = \int_{a}^{b} f(x) \, dx = \Cd_{a}^{b} f = \Cg_{a}^{b} f.
    \end{equation}
\end{defr}

\begin{twier}{Kryterium całkowalności}
    $f: [a, b] \to \R$ jest całkowalna wtedy i tylko wtedy, gdy: \begin{equation}
        \forall_{\eps > 0} \exists_{\pi \preceq [a, b]}: \Sg(f, \pi) - \Sd(f, \pi) \leq \eps.
    \end{equation}
\end{twier}

\begin{dow}
    \paragraph{$\impliedby$} Jeśli spełniony jest warunek: \begin{equation*}
        \forall_{\eps > 0} \exists_{\pi \preceq [a, b]}: \Sg(f, \pi) - \Sd(f, \pi) \leq \eps,
    \end{equation*} to skoro: \begin{equation*}
        \Sd(f, \pi) \leq \Cd_{a}^{b} f \leq \Cg_{a}^{b} f \leq Sg(f, \pi),
    \end{equation*} gdyż skoro: dla każdych $\pi_1, \pi_2$: \begin{equation*}
        \Sd(f, \pi_1) \leq S(f, \pi_2), 
    \end{equation*} to na pewno: \begin{equation}
        \Cd_{a}^{b} f = \sup_{\pi_1}  \Sd(f, \pi_1) \leq \inf_{\pi_1}  S(f, \pi_2) = \Cg_{a}^{b}f.
    \end{equation} Dostajemy, więc: \begin{equation*}
        \Cg_{a}^{b} f - \Cd_{a}^{b} f \leq Sg(f, \pi) - Sd(f, \pi) \leq \eps \forall_{\eps > 0}.
    \end{equation*} Jasne, że zachodzić to może tylko wtedy, gdy całka dolna jest równa całce górnej - $f$ jest więc całkowalna.

    \paragraph{$\implies$} Mamy:  $\Cg_{a}^{b} f = \Cd_{a}^{b} f$. Z deficji supremum i infimum wynika, że: \begin{equation*}
        \exists_{\pi_1}: \Sg(f, \pi_1) - \Cg_{a}^{b} f \leq \half \eps
    \end{equation*} oraz: \begin{equation*}
        \exists_{\pi_2}:  \Cd_{a}^{b} f - \Sd(f, \pi_2)\leq \half \eps.
    \end{equation*} Niech $\pi \preceq \pi_1, \pi_2$. Dla tego podziału zachodzić będą oczywiście obie z tych nierówności ($\Sg (f, \pi) \leq \Sg(f, \pi_1)$ i $\Sd (f, \pi) \geq \Sd(f, \pi_2)$). Zatem, dodawszy je: \begin{equation*}
        \Sg(f, \pi) - \Cg_{a}^{b} f + \Cd_{a}^{b} f - \Sd(f, \pi) = \Sg(f, \pi)  - \Sd(f, \pi) \leq \eps.
    \end{equation*}
\end{dow}


\begin{twier}{Złożenie funkcji ciągłej z funkcją całkowalną jest całkowalne}
    Niech $F: [a, b] \to \R$ będzie funkcją całkowalną (więc i ograniczoną). Niech $F: \ol{f([a, b])} \to \R$ jest ciągła (dziedzina $F$ to domknięcie obrazu $f$). \\
    Wtedy $F \circ f: [a, b] \to \R$ jest całkowalna.
\end{twier}



\begin{dow}{}
    Zbiór $X = \ol{f([a, b])}$ jest domknięty i ograniczony w $\R$, więc jest zwarty. Zatem $F$, jako ciągła na zbiorze zwartym, jest jednostajnie ciągła: \begin{equation*}
        \forall_{\eps} \exists_{\delta_\eps >0 } | s - t| \leq \eps \implies | F(s) - F(t) | \leq \frac{\eps}{b - a + 2 \sup_{X} |F|}  = \eps'. 
    \end{equation*}  Wyrażenie po prawej ma sens, gdyż $F$ jest ograniczona. Przyjmijmy w tym warunku, że $\delta_\eps < \eps'$, co zawsze możemy zrobić. Ponadto, skoro $f$ - całkowalna, to: \begin{equation*}
        \exists_\pi: \Sg(f, \pi) - \Sd(f, \pi) \leq \delta_\eps^2.
    \end{equation*} Niech $\pi = (t_0, ..., t_n)$. Ponadto niech $\sup_{[t_{i-1}, t_{i} ]} f = M_i$, $\inf_{[t_{i-1}, t_{i} ]} f = m_i$,  $\sup_{[t_{i-1}, t_{i} ]} F \circ f = M_i'$, $\inf_{[t_{i-1}, t_{i} ]} F \circ f = m_i'$. Z właności $f$ i $F$ wiemy, że są to wszystko liczby rzeczywiste. Zatem: \begin{equation*}
        \Sg(f, \pi) - \Sd(f, \pi)  = \sum_{i = 1}^{n} (M_i - m_i) (t_i - t_{i-1}).
    \end{equation*} Rozbijmy indeksy $\{1, ..., n\} = A \sqcup B$, gdzie $A = \{i \, | \, M_i - m_i \leqslant \delta_\eps \}$, $B = \{i \, | \, M_i - m_i > \delta_\eps \}$. Zatem: \begin{equation*}
        \Sg(F \circ f, \pi) - \Sd(F \circ f, \pi)  = \sum_{i \in A} (M_i' - m_i') (t_i - t_{i-1}) + \sum_{i \in B} (M_i' - m_i') (t_i - t_{i-1})
    \end{equation*} Jeśli $i \in A$ i $x, y \in [t_{i -1}, t_i]$, to $m_i \leq f(x), f(y) \leq M_i$, więc $|f(x) - f(y)| \leq M_i - m_i \leq \delta_\eps$. Zatem z jednostajnej ciągłości $F$: $|F(f(x)) - F(f(y))|  \leq \eps'$. W szczególności więc $|M_i' - m_i'| \leq \eps$, gdyż przy przejściu do supremum i infimum nierówność się zachowuje. Tak więc: \begin{equation*}
        \sum_{i \in A} (M_i' - m_i') (t_i - t_{i-1}) \leq \sum_{i \in A} \eps' (t_i - t_{i-1}) \leq \eps' (b- a).
    \end{equation*} Co do drugiej sumy, mamy: \begin{equation*}
        \delta_\eps \sum_{i \in B} (t_i - t_{i-1}) < \sum_{i \in B} (M_i - m_i)  (t_i - t_{i-1}) \leq \sum_{i = 1}^{n} (M_i - m_i)  (t_i - t_{i-1}) = \Sg(f, \pi) - \Sd(f, \pi) \leq \delta_\eps^2.
    \end{equation*} Stąd: \begin{equation*}
        \sum_{i \in B} (t_i - t_{i-1}) < \delta_\eps \leq \eps'
    \end{equation*} Zatem: \begin{equation*}
        \sum_{i \in B} (M_i' - m_i') (t_i - t_{i-1}) \leq (\sup_{X} F - \inf_{X} F) \sum_{i \in B} (t_i - t_{i-1}) \leq 2 \sup_{X} |F| \eps' < \oo,
    \end{equation*} bo $F$ jest ograniczona. Ostatecznie: \begin{equation*}
        \Sg(F \circ f, \pi) - \Sd(F \circ f, \pi)  =  \sum_{i \in A} (M_i' - m_i') (t_i - t_{i-1}) + \sum_{i \in B} (M_i' - m_i') (t_i - t_{i-1}) \leq \eps'( b- a + 2 \sup_X |F|) = \eps.
    \end{equation*} Konstrukcje przeprowadziliśmy dla dowolnego $\eps$, zatem $F \circ f$ spełnia kryterium całkowalności, więc jest całkowalna.
\end{dow}

\paragraph{Przykład} Zbadajmy funckję $\id: [a, b] \to \R$, t.ż. $\id(x) = x$. Jest to funkcja całkowalna na dowolnym przedziale. Rzeczywiście, niech $\pi = (a + \frac{1}{n}(b - a) \, | \, 0 \leq i \leq n)$. Wtedy $\sup_{\przedz{t}{i}} \id = (b - a) \frac{i +1}{n} + a$ i $\inf_{\przedz{t}{i}} \id = (b - a) \frac{i}{n} + a$. Zatem: \begin{equation}
    \Sg(\id, \pi) - \Sd(\id, \pi) = \sum_{i = 1}^{n} \left(\frac{b- a}{n}\right)^2 = \frac{(b - a)^2}{n}.
\end{equation} Wielkość tę można uczynić dowolnie małą dla dużych $n$, więc $\id$ spełnia kryterium całkowalności.

\begin{obs}{Funkcje ciągłe są całkowalne}
    Niech $f: [a, b] \to \R$. Wtedy $f$ jest całkowalna jako złożenie funkcji ciągłej z funkcją całkowalną: $f = f \circ \id$.
\end{obs}


\begin{twier}{Liniowość całki}
    Niech $f, g: [a, b] \to \R$ - całkowalne i $\alpha, \beta \in \R$. Wtedy $\alpha f + \beta g: [a, b] \to \R$ jest całkowalna i: \begin{equation}
        \int_{a}^{b} \alpha f + \beta g = \alpha \int_{a}^{b} f + \beta \int_{a}^{b} g.
    \end{equation} Inaczej mówiąc - przestrzeń funkcji całkowalnych na $[a, b]$ jest przestrzenią liniową, a wzięcie całki jest na tej przestrzenii formą liniową.
\end{twier}

\begin{dow}{}
    \paragraph{Jednorodność} Niech $\alpha \in \R$ i $f$ - całkowalna. Wtedy, gdy $\alpha > 0$: \begin{equation*}
        \Sg( \alpha f, \pi) = \alpha \Sg(  f, \pi)
    \end{equation*} oraz: \begin{equation*}
        \Sd( \alpha f, \pi) = \alpha \Sd(  f, \pi)
    \end{equation*} z jednorodności $\sup$ i $\inf$ dla skalarów dodatnich. Zatem, z całkowalności $f$: \begin{equation}
        \exists_{\pi}:  \Sg( f, \pi) - \Sd( f, \pi) \leq \frac{\eps}{\alpha} \implies \exists_{\pi}:  \Sg( \alpha f, \pi) - \Sd( \alpha f, \pi) \leq \eps,
    \end{equation} czyli spełniony jest warunek całkowalności dla $\alpha f$. \\
    Gdy $\alpha = 0$, to $\alpha f := 0$, a funckcja zerowa jest trywialnie całkowalna. \\
    Gdy $\alpha < 0$ mamy: \begin{equation*}
        \Sg( \alpha f, \pi) = \alpha \Sd(  f, \pi)
    \end{equation*} oraz: \begin{equation*}
        \Sd( \alpha f, \pi) = \alpha \Sg(  f, \pi),
    \end{equation*} z tego, że $\sup -f = - \inf f$ i vice versa. Dalej dowód przeprowadza się analogicznie.
    
    \paragraph{Addytywność} Z całkowalności $f, g$ wynika, że dla dowolnego $\eps$: \begin{equation*}
        \exists_{\pi_1}: \Sg(f, \pi_1) - \Sd(f, \pi_1) \leq \half \eps
    \end{equation*} i podobnie: \begin{equation*}
        \exists_{\pi_2}: \Sg(g, \pi_2) - \Sd(g, \pi_2) \leq \half \eps
    \end{equation*} Weźmy $\pi = (t_0, ..., t_n) \preceq \pi_1, \pi_2$. Wtedy powyższe nierówności tym bardziej są spełnione dla podziału $\pi$: \begin{equation*}
        \Sg(f, \pi) - \Sd(f, \pi) + \Sg(g, \pi) - \Sd(g, \pi) \leq \eps.
    \end{equation*} Mamy też: \begin{equation*}
        \Sg(f + g, \pi) = \sum_{i = 1}^{n} \sup_{\przedz{t}{i}} (f + g) \leq  \sum_{i = 1}^{n} \sup_{\przedz{t}{i}} f + \sup_{\przedz{t}{i}} g = \Sg(f, \pi) + \Sg(g, \pi).
    \end{equation*} Identycznie: \begin{equation*}
        \Sd(f + g, \pi) \geq \Sd(f, \pi) + \Sd(g, \pi).
    \end{equation*} Zatem: \begin{equation}
        \Sg(f + g, \pi) - \Sd(f + g, \pi) \leq \Sg(f, \pi) - \Sd(f, \pi) + \Sg(g, \pi) - \Sd(g, \pi) \leq \eps.
    \end{equation} Widzimy więc, żę $f +g$ spełnia kryterium całkowalności, więc jest całkowalna.\\
    Z tego wynika, że $\sup_{\pi} \Sd(f + g, \pi) = \inf_{\pi} \Sg(f + g, \pi) = \int_{a}^{b} (f + g).$ Z całkowalności $f, g$ wynika też, że dla dowolnego $\eps$: \begin{equation*}
        \exists_{\pi}: \Sg(f, \pi) \leq \int_{a}^{b} f + \half \eps \, \wedge \, \Sg(g, \pi) \leq \int_{a}^{b} g + \half \eps.
    \end{equation*} Tutaj podobnie wybieramy osobno podziały dla $f$ i $g$ a następnie znajdujemy podział drobniejszy od ich obu. Wtedy: \begin{equation*}
        \int_{a}^{b} f + g \leq \Sg(f + g, \pi) \leq \Sg(f , \pi) + \Sg(g, \pi) \leq  \int_{a}^{b} f +  \int_{a}^{b} g + \eps.
    \end{equation*} Skoro nierówność ta zachodzi dla wszystkich $\eps > 0$, to musi też być: \begin{equation*}
        \int_{a}^{b} f + g \leq \int_{a}^{b} f +  \int_{a}^{b} g.
    \end{equation*} Zastosujmy teraz powyższą nierówność do pary $(-f, -g)$ i skorzystajmy z jednorodności całki: \begin{equation}
        -\int_{a}^{b} f + g = \int_{a}^{b} -f - g \leq \int_{a}^{b} -f   + \int_{a}^{b} -g =-(\int_{a}^{b} f   + \int_{a}^{b} g).
    \end{equation} Łącząc otrzymane nierówności, mamy: \begin{equation*}
        \int_{a}^{b} f + g = \int_{a}^{b} f   + \int_{a}^{b} g
    \end{equation*}

\end{dow}



\begin{twier}{}
    Niech $f, g: [a, b] \to \R$ będą funkcjami całkowalnymi. Wtedy $f \cdot g: [a, b] \to \R$ także jest całkowalna.
\end{twier}

\begin{dow}{}
    Niech $F: [a, n] \ni x \mapsto x^2$. Jest to funkcja ciągła. Wtedy $ f \cdot g = \frac{1}{4}( F \circ (f + g) - F \circ (f - g))$ jest całkowalna na podstawie udowodnionych już własności. 
\end{dow}


\paragraph{Przykład - funkcja Dirichleta}
    Niech \begin{equation}
        Z (x) = \begin{cases}
            1 & x \in \Q \\
            0 & x \notin \Q
        \end{cases}.
    \end{equation} Funkcja ta nie jest całkowalna na żadnym przedziale, gdyż dla każdego podziału $\pi \preceq [a, b]$, mamy: \begin{equation}
        \Sg(Z, \pi ) = b - a \neq 0 = \Sd(Z, \pi),
    \end{equation} gdyż w każdym przedziale o niezerowej długości znajdują się zarówno liczby wymierne jak i niewymierne.

\begin{twier}{}
    Niech $f, g: [a, b]  \to \R$ - całkowalne i $\forall_{x \in [a, b]}: f(x) \leq g(x)$ - lub pisząc prościej: $f \leq g$. Wtedy: \begin{equation}
        \int_{a}^{b} f \leq \int_{a}^{b} g.
    \end{equation}
\end{twier}

\begin{dow}{}
    Dowód niesamowicie prosty - skoro $f \leq g$, to $\Sg(f, \pi) \leq \Sg(g, \pi)$ dla każdego podziału $\pi$. Zatem oczywiście: \begin{equation*}
        \int_{a}^{b} f = \inf_{\pi} \Sg(f, \pi) \leq \inf_{\pi} \Sg(g, \pi) = \int_{a}^{b} g.
    \end{equation*}
\end{dow}


\begin{twier}{}
    Niech $f: [a, b]  \to \R$ - całkowalna. Wtedy $|f|$ jest także całkowalna i: \begin{equation}
        \big| \int_{a}^{b} f \big| \leq \int_{a}^{b} \big| f \big|.
    \end{equation}
\end{twier}


\begin{dow}{}
    $|f|$ musi być całkowalna, jako złożenie funkcji ciągłej $F(x) = |x|$ i funkcji całkowalnej $f$. Ponadto istnieje taka liczba $\sigma = \pm 1$: \begin{equation*}
        \sigma \int_{a}^{b} f = \big| \int_{a}^{b} f \big|.
    \end{equation*} Ale, korzystając z poprzedniego rezultatu:\begin{equation*}
        \big| \int_{a}^{b} f \big| = \sigma \int_{a}^{b} f =  \int_{a}^{b} \sigma f \leq \int_{a}^{b} \big| \sigma f \big| = \int_{a}^{b} \big| f \big|,
    \end{equation*} gdyż $\sigma f \leq |f|$.
\end{dow}

\begin{obs}[label=obs:punkt-roznica-calek]{} 
    \begin{itemize}
        \item Niech $f, g: [a, b]  \to \R$. Ponadto, niech $f = g$ na $[a, b] \backslash S$, gdzie $|S| < \oo$ - tj. $f$ i $g$ zgadzają się z sobą na całym przedziale poza skończoną  liczbą punktów. \\
        Wtedy $f$ jest całkowalna wtedy i tylko wtedy, gdy $g$ jest całkowalna i jeśli to zachodzi, to: \begin{equation*}
            \int_{a}^{b} f = \int_{a}^{b} g.
        \end{equation*}
        \item Niech $f: [a, b] \to \R$ - całkowalna i niech $[c, d] \se [a, b]$. Wtedy $f \big|_{[c, d]}$ także jest całkowalna.
        \item Zdefiniujmy funkcję charakterystyczną zbioru $A$: \begin{equation}
            \chi_A(x) = \begin{cases}
                1 & x \in A \\
                0 & x \notin A
            \end{cases}.
        \end{equation} 
        Wtedy, jeśli $f: [a, b] \to \R$ jest całkowalna, to dla $[c, d] \se [a, b]$ mamy: \begin{equation}
            \int_{a}^{b} \chi_{[c, d]}\cdot  f = \int_{c}^{d} f \big|_{[c, d]} = \int_{c}^{d} f.
        \end{equation} Ostatnie wyrażenie jest wygodniejszym zapisem tejże wielkości.
    \end{itemize}
\end{obs}


\begin{twier}{}
    Niech $f: [a, b] \to \R$ - całkowalna i $c \in ]a, b[$. Wtedy: \begin{equation}
        \int_{a}^{b} f = \int_{a}^{c} f + \int_{c}^{b} f.
    \end{equation}
\end{twier}

\begin{dow}{}
    \begin{equation}
        \int_{a}^{b} f = \int_{a}^{b} (\chi_{[a, c] } f + \chi_{[c, b]}  f) =  \int_{a}^{b} \chi_{[a, c] } f + \int_{a}^{b} \chi_{[c, b]}  f = \int_{a}^{c} f  + \int_{c}^{d} f,
    \end{equation} bo $f$ i $\chi_{[a, c] } f + \chi_{[c, b]}  f$ różnią się tylko w jednym punkcie ($x = c$). 
\end{dow}


\begin{defr}{Wypunktowanie}
    Niech $\pi = (t_0, ..., t_n)$ będzie podziałem $[a, b]$. Wtedy \tb{wypunktowaniem} $\pi$ nazywamy ciąg skończony: \begin{equation}
        \Xi = (\xi_1, ..., \xi_n),
    \end{equation} t.ż. $\xi_i \in \przedz{t}{i}$. \\
    Ponadto, jeśli $f: [a, b] \to \R$, to \tb{sumą wypunkowaną} podziału $\pi$ i wypunktowania $\Xi$ nazywamy: \begin{equation}
        S(f, \pi, \Xi) = \sum_{i = 1}^{n} (t_i - t_{i- 1}) f(\xi_i).
    \end{equation} Średnicą podziału $\pi$ nazwiemy liczbę: \begin{equation}
        \text{Diam}(\pi) = \max_{1 \leq i \leq n} (t_i - t_{i-1}).
    \end{equation}
\end{defr}



\begin{twier}[label=twier:rozbicie-calki]{}
    Niech $f \in C([a, b])$ (funkcje ciągłe na $[a, b]$) i $\ciag{\pi}$ będzie ciągiem podziałów $[a, b]$, t.ż.:\begin{equation}
        \text{Diam}(\pi_n) \to 0.
    \end{equation} Niech $\ciag{\Xi}$ będzie ciągiem wypunktowań $\pi_n$. \\
    Wtedy: \begin{equation}
        S(f, \pi_n, \Xi_n) \xrightarrow[]{n \to \oo} \int_{a}^{b} f
    \end{equation}
\end{twier}

\begin{dow}{}
    Zacznijmy od tego, że $f$ jako funkcja ciągła jest całkowalna. Ponadto, $f$ jako ciągła na zwartym przedziale jest jednostajnie ciągła: \begin{equation}
        \forall_{\eps > 0} \exists_{\delta_\eps}: |s - t| \leq \delta_\eps \implies |f(s) - f(t)| \leq \eps.
    \end{equation} Niech $\pi$ będzie takim podziałem, że $\text{Diam}(\pi) \leq \delta_{\frac{\eps}{b - a}}$. Wtedy: \begin{equation}
        \Sg(f, \pi) - \Sd(f, pi) = \sum_{i = 1}^{n} (\sup_{\przedz{t}{i}} f - \inf_{\przedz{t}{i}} f) (t_i - t_{i -1}) \leq \sum_{i = 1}^{n} \frac{\eps}{b - a}(t_i - t_{i -1}) = \eps,
    \end{equation} co wynika z tego, że skoro $(t_i - t_{i -1}) \leq \delta_{\frac{\eps}{b - a}}$, to wartości $f$ na tym przedziale są nie dalej niż $\frac{\eps}{b - a}$, więc także jest tak dla supremum i infimum $f$ na tym przedziale. \\
    Ponadto, dla dowolnego wypunktowania $\Xi$ podziału $\pi$: \begin{equation}
        \Sd(f, \pi) =  \sum_{i = 1}^{n} \inf_{\przedz{t}{i}} f (t_i - t_{i -1}) \leq \sum_{i = 1}^{n} f(\xi_i) (t_i - t_{i -1})  = S(f, \pi, \Xi)
    \end{equation} oraz: \begin{equation}
        S(f, \pi, \Xi) = \sum_{i = 1}^{n} f(\xi_i) (t_i - t_{i -1}) \leq \sum_{i = 1}^{n} \sup_{\przedz{t}{i}} f (t_i - t_{i -1}) =  \Sg(f, \pi),
    \end{equation} czyli: \begin{equation}
        \Sd(f, \pi) \leq  S(f, \pi, \Xi)  \leq \Sg(f, \pi).
    \end{equation} Jednak zachodzi też: \begin{equation}
        \Sd(f, \pi) \leq  \int_{a}^{b} f \leq \Sg(f, \pi),
    \end{equation} Zatem: \begin{equation}
        \big| S(f, \pi, \Xi)  - \int_{a}^{b} f \big| \leq \Sg(f, \pi) - \Sd(f, \pi) \leq \eps.
    \end{equation} Jśli dla ustalonego $\eps > 0$ w naszym ciągu $\ciag{\pi}$, wybierzemy takie $N$, że dla $n \geq N$: \begin{equation}
        \text{Diam}(\pi_n) \leq \delta_{\frac{\eps}{b- a}},
    \end{equation} to: \begin{equation}
        \Sg(f, \pi_n) - \Sd(f, \pi_n) \leq \eps 
    \end{equation} i: \begin{equation}
        \big| S(f, \pi_n, \Xi_n)  - \int_{a}^{b} f \big| \leq \eps
    \end{equation} dla $n \geq N$. Zatem rzeczywiście: \begin{equation}
        S(f, \pi_n, \Xi_n) \to  \int_{a}^{b} f.
    \end{equation}
\end{dow}




\begin{defr}{}
    Dla $a > b$ definiujemy: \begin{equation}
        \int_{a}^{b} f := -\int_{b}^{a} f.
    \end{equation} Ponadto: \begin{equation}
        \int_{a}^{a} f := 0.
    \end{equation} Zauważmy, że wtedy Twierdzenie \ref*{twier:rozbicie-calki} zachodzi dla dowolnego ułożenia $a, b, c$. 
\end{defr}

\begin{twier}[label=twier:ZTRCI]{Zasadnicze Twierdzenie Rachunku Całkowego I}
    Niech $f: [a, b] \to \R$ -  całkowalna. Ponadto zdefiniujmy $F: [a, b] \to \R$: \begin{equation}
        F(x) = \int_{a}^{x} f.
    \end{equation} Wiemy, że powyższe wyrażenie ma sens dla $x \in [a, b]$.\\
    Wtedy: \begin{itemize}
        \item $F$ jest ciągła,
        \item jeśli $f$ jest ciągła w $x_0$, to $F$ jest różniczkowalna tamże i: \begin{equation}
            F'(x_0) = f(x_0).
        \end{equation}
    \end{itemize} 
\end{twier}



\begin{dow}{Dowód}
    \begin{itemize}
        \item     Skoro $f$ jest całkowalna, to musi być ograniczona: \begin{equation*}
            \exists_M \forall_x: |f(x)| \leq M.
        \end{equation*} Zatem dla $ x > y$: \begin{equation}
            |F(x) - F(y)| = \big|\int_{a}^{x} f - \int_{a}^{y} f\big| = \big| \int_{x}^{y} f\big|\leq \int_{x}^{y} |f| \leq M |x - y|,
        \end{equation} czyli $F$ jest Lipschitzowska, a więc i ciągła.

        \item Niech $f$ będzie ciągła w $x_0$. Obliczmy iloraz różniczowy $F$:\begin{equation}
            \frac{F(x_0 + h) - F(x_0)}{h} = \frac{1}{h} \int_{x_0}^{x_0 + h} f = \frac{1}{|h|} \int_{c}^{d} f,
        \end{equation} gdzie: \begin{equation}
            [c, d] = \begin{cases}
                [x_0, x_0 + h] & h > 0 \\
                [x_0 + h, x_0] & h < 0
            \end{cases}.
        \end{equation}
        Skoro $f$ jest ciągła w $x_0$, to: \begin{equation}
            \forall_{\eps > 0} \exists_{\delta_\eps}: | t - x_0| \leq \delta_\eps \implies |f(t) - f(x_0)| \leq \eps. 
        \end{equation} Wtedy dla $|h| \leq \delta_\eps$ i $t \in [c, d]$: \begin{equation}
            f(x_0) - \eps \leq f(t) \leq f(x_0) + \eps.
        \end{equation} Zatem, całkując obustronnie: \begin{equation}
            |h| (f(x_0) - \eps) \leq  \int_{c}^{d} f \leq |h| (f(x_0) + \eps),
        \end{equation} i dalej: \begin{equation}
            f(x_0) - \eps \leq \frac{F(x_0 + h) - F(x_0)}{h} \leq f(x_0) + \eps,
        \end{equation} dla $|h| \leq \delta_\eps$. Zatem: \begin{equation}
            \forall_{\eps > 0} \exists_{\delta_\eps}: |h| \leq \delta_\eps \implies \big|  \frac{F(x_0 + h) - F(x_0)}{h} - f(x_0) \big| \leq \eps.
        \end{equation} Jest to inne wyrażenie równości: \begin{equation}
            F'(x_0) = \lim_{h \to 0} \frac{F(x_0 + h) - F(x_0)}{h} = f(x_0).
        \end{equation}
    \end{itemize}

\end{dow}

\paragraph{} Zauważmy, że skoro: \begin{equation}
    \int_{x_0}^{x} f = -\int_{a}^{x_0} f + \int_{a}^{x}f,
\end{equation} gdzie $\int_{a}^{x_0} f = $ const., to zdefiniowanie $F$ jako całki od $a$ lub od $x_0$ przesuwa ją jedynie i stałą i nie zmienia jej własności.

\begin{defr}{Funkcja pierwotna}
    Niech $f$ będzie daną funkcją. Jeśli $F$ jest taką funkcją, że $F' = f$, to $F$ nazywamy \tb{funkcją pierwotną} do $f$.
\end{defr}

\begin{obs}{}
    Dla każdej funkcji ciągłej $f$ na $[a, b]$ istnieje funkcja do niej pierwotna: \begin{equation}
        F(x) = \int_{a}^{x} f.
    \end{equation}
\end{obs}

\begin{twier}[label=twier:ZTRCII]{Zasadnicze Twierdzenie Rachunku Całkowego II}
    Jeśli $f: [a, b] \to \R$ jest całkowalna, a $F: [a, b] \to \R$ jest ciągła na $[a, b]$ i różniczkowalna na $]a, b[$ oraz ponadto: \begin{equation}
        \forall_{x \in ]a, b[ }: F'(x) = f(x),
    \end{equation} to wtedy: \begin{equation}
        \int_{a}^{b} f = F(b) - F(a).
    \end{equation}
\end{twier}

\begin{dow}{}
    Weźmy $\eps > 0$. Niech $\pi$ będzie takim podziałem $[a, b]$, że: \begin{equation}
        \Sg(f, \pi) - \Sd(f, \pi) \leq \eps.
    \end{equation} Niech $\pi = (t_0, ..., t_n)$. Niech $\Xi = (\xi_1, ..., \xi_n)$ będzie wypunktowaniem $\pi$ t.ż.:\begin{equation}
        \frac{F(t_i) - F(t_{i - 1})}{ t_i - t_{i -1} } = F'(\xi_i) = f(\xi_i).
    \end{equation} $\xi_i \in ]t_i, t_{i-1}[$ spełniające powyższy warunek będzie istnieć na mocy Twierdzenia Lagrange'a. Skoro: \begin{equation}
        \Sg(f, \pi) \leq S(f, \pi, \Xi) \leq \Sd(f, \pi)
    \end{equation} oraz: \begin{equation}
        \Sg(f, \pi) \leq \int_{a}^{b} f \leq \Sd(f, \pi),
    \end{equation} to: \begin{equation}
        \big| \int_{a}^{b} f - S(f, \pi, \Xi) \big| \leq \eps.
    \end{equation} Mamy też: \begin{equation}
        S(f, \pi, \Xi) = \sum_{i = 1}^{n} f(\xi_i) ( t_i - t_{i -1}) = \sum_{i = 1}^{n} F'(\xi_i) ( t_i - t_{i -1}) =  \sum_{i = 1}^{n} F(t_i) - F(t_{i - 1}) = F(b) - F(a).
    \end{equation} Zatem: \begin{equation}
        \forall_{\eps > 0}: \big| \int_{a}^{b} f - (F(b) - F(a)) \big| \leq \eps.
    \end{equation} Wielkości te muszą więc być równe sobie: \begin{equation}
        \int_{a}^{b} f  = F(b) - F(a).
    \end{equation}
\end{dow}

\begin{twier}{Całkowanie przez części}
    Niech $F,G \in C([a, b])$ - różniczkowalne na $]a, b[$. Niech ponadto $f = F'$ i $g = G'$ - całkowalne na $[a, b]$ (zauważmy, że całkowalność $f, g$ nie zależy od ich wartości w $a$ i $b$, por. Obs. \ref*{obs:punkt-roznica-calek}). \\
    Wtedy: \begin{equation}
        \int_{a}^{b} f G = G \cdot F \big|_{a}^{b} - \int_{a}^{b} F g.
    \end{equation}
\end{twier}

\begin{dow}{}
    $F \cdot G$ jest funkcją pierwotną do $Fg + fG$. Zatem $Fg + fG$ jest całkowalna (jako suma i iloczyn f. całkowalnych) i: \begin{equation}
        \int_{a}^{b} (Fg + f G) = G \cdot F \big|_{a}^{b},
    \end{equation} co wynika z Tw. \ref*{twier:ZTRCII}. Korzystając z liniowości całki i z tego, że każda az funkcji $Fg$ i $fG$ jest całkowalna, dostajemy tezę.
\end{dow}


\begin{twier}{Całkowanie przez podstawienie}
    Niech $\varphi \in C([a, b])$ będzie ściśle rosnącą funkcją różniczkowalną na przedziale $]a, b[$. Niech $[c, d] = \varphi([a, b])$ (wiemy że obrazem zwartego przedziału przy funkcji ciągłej będzie zwarty przedział). Niech $f \in C([c, d])$. \\
    Wtedy funkcja $(f \circ \varphi) \varphi'$ t.ż.: \begin{equation}
        [a, b] \ni x \mapsto f(\varphi(x)) \varphi'(x) \in R, 
    \end{equation} jest całkowalna i :\begin{equation}
        \int_{a}^{b} (f \circ \varphi) \varphi' dx = \int_{c}^{d} f(y) dy.
    \end{equation} Zauważmy, że w punktach $a, b$ funkcja $(f \circ \varphi) \varphi'$ nie jest dobrze określona, jednak możemy przyjąć tam dowolne wartości, co nie zmieni całkowalności.\\
    Uwaga! W twierdzeniu tym zamiast ciągłości $f$ przyjąć możemy słabsze założenia: \begin{itemize}
        \item $f$ ma funkcję pierwotną,
        \item $(f \circ \varphi) \varphi '$ jest całkowalna.
    \end{itemize}
\end{twier}

\begin{dow}{}
    $f \circ \varphi$ jest całkowalna jako złożenie dwu funkcji ciągłych. Po pomnożeniu przez $\varphi'$, całkowalną na mocy założeń, otrzymamy funkcję całkowalną $(f\circ \varphi) \cdot \varphi'$. Niech: \begin{equation*}
        F(y) = \int_{c}^{y} f dy,
    \end{equation*}  dla $y \in [c, d]$. $F$ jest różniczkowalna i $F' = f$. Jeśli przyjmujemy ogólniejsze założenia, to $F$ jest funkcją pierwotną do $f$ taką, że $F(c)$. Taka zawsze istnieje, gdyż jeśli $G$ jest dowolną f. pierwotną do $f$, to $F = G - G(c)$ także nią jest. Wtedy także $F \circ \varphi$ jest różniczkowalna na na $]a, b[$ i:\begin{equation*}
        (F \circ \varphi)' = (F' \circ \varphi) \cdot \varphi' = (f \circ \varphi) \cdot \varphi'.
    \end{equation*} Zatem $F \circ \varphi$ jest funkcją pierwotną do $(f \circ \varphi) \varphi'$ na $]a, b[$. Zatem, zgodnie z \ref*{twier:ZTRCII}: \begin{equation*}
        \int_{a}^{b} f(\varphi(x)) \varphi'(x) dx = (F \circ \varphi)(b) - (F \circ \varphi)(a),
    \end{equation*} ale, skoro $\varphi$ jest ściśle rosnąca, to $\varphi(a) = c$ i $\varphi(b) =d$, więc:\begin{equation*}
        \int_{a}^{b} f(\varphi(x)) \varphi'(x) dx = F(d) - F(c).
    \end{equation*} Ale $F$ jest funkcją pierwotną do $f$, więc: \begin{equation*}
        \int_{c}^{d} f(y) dy  = F(d) - F(c).
    \end{equation*} Porównując dwa poprzednie równania, mamy tezę: \begin{equation*}
        \int_{a}^{b} f(\varphi(x)) \varphi'(x) dx =  \int_{c}^{d} f(y) dy.
    \end{equation*}
\end{dow}
% TODO UWAGA! na f. malejace


\subsection{Funkcje logarytmiczna i wykładnicza}

\begin{defr}{Logarytm naturalny}
    Dla $x > 0$ zdefiniujmy funkcję $\log x: ]0, \oo[ \to \R$: \begin{equation}
        \log x = \int_{1}^{x} \frac{1}{t} dt.
    \end{equation} Na mocy Tw. \ref*{twier:ZTRCI}, $\log$ jest ciągły i różniczkowalny na całej dziedzinie oraz: \begin{equation}
        (\log x )' = \frac{1}{x}.
    \end{equation} Z tego widać też, że $\log$ jest funkcją ściśle rosnącą.
\end{defr}

\begin{twier}{}
    Niech $x, y \in ]0, \oo[$. Wtedy: \begin{equation*}
        \log(xy) = \log x + \log y.
    \end{equation*}
\end{twier}

\begin{dow}{}
    \begin{equation*}
        \log (xy) = \int_{1}^{xy} \frac{1}{t} dt = \int_{1}^{y} \frac{1}{t} dt + \int_{y}^{xy} \frac{1}{t} dt = \log y + \int_{y}^{xy} \frac{1}{t} dt.
    \end{equation*} Weźmy pod lupę tę ostatnią całkę i użyjmy tu podstawienia: $\varphi(t) = yt$, $\varphi'(t) = y > 0$. Wtedy: \begin{equation*}
        \int_{y}^{yt} \frac{1}{t} dt = \int_{\varphi^{-1}(y)}^{\varphi^{-1}(xy)} \frac{1}{\varphi(t)} \varphi'(t) dt = \int_{1}^{x} \frac{1}{t} dt = \log x.
    \end{equation*} Podstawiając powyższą wartość do wyjściowego równania otrzymujemy tezę.
\end{dow}

\begin{twier}[label=twier:log-potega]{}
    Niech $r \in \Q$, $x \in ]0, \oo[$. Wtedy: \begin{equation*}
        \log (x ^ r) = r \log x.
    \end{equation*}
\end{twier}

\begin{dow}{}
    Ponownie, używając łatwego podstawienia: \begin{equation*}
        \log(x^r) = \int_{1}^{x^r} \frac{1}{t} dt = \int_{1}^{x} \frac{r t^{r-1}}{t^r} dt = r \log x.
    \end{equation*} Użyliśmy tu podstawienia: $\varphi (x) = x ^r$.
\end{dow}

\paragraph{} Zauważmy, że $\log$ jest ciągły na przedziale spójnym, jego obrazem więc będzie przedział. Zeby wyznaczyć jego postać, zauważmy, że: \begin{equation}
    \log 2 = \int_{1}^{2} \frac{1}{t} dt \geq \frac{1}{2} > 0
\end{equation} i ogólnie $\log x \geq \frac{x - 1}{x}$, dla $x \geq 1$. Stąd: \begin{equation}
    \log (2 ^ n) = n \log 2 \geq \half n \xrightarrow[]{n \to \oo} \oo.
\end{equation} Stąd z monotoniczności $\log$: \begin{equation}
    \lim_{x \to \oo} \log x = \oo.
\end{equation} Podobnież: \begin{equation}
    \log (2 ^ {-n}) = -n \log 2 \leq -\half n \xrightarrow[]{n \to \oo} -\oo,
\end{equation} czyli: \begin{equation}
    \lim_{x \to 0} \log x = -\oo.
\end{equation}
Zatem przeciwobrazem $\log$ jest $\R$. Stąd i z monotoniczności (implikującej różnowartościowość) i ciągłośći (pociągającej suriektywność) mamy:

\begin{obs}{}
    $\log: ]0, \oo[ \to \R$ jest różniczkowalną bijekcją $]0, \oo[$ na $\R$.
\end{obs}


Wynika stąd, że istnieje funkcja odwrotna do $\log$ i ona także będzie różniczkowalną bijekcją.

\begin{defr}{Funkcja wykładnicza}
    Funkcję: \begin{equation}
        \exp(x): \R \to ]0, \oo[,
    \end{equation} będącą funkcją odwrotną do $\log$ nazywamy \tb{funkcją wykładniczą}. Jest to różniczkowalna bijekcja.
\end{defr}

\begin{twier}{Pochodna funkcji wykładniczej}
    \begin{equation}
        (\exp)' = \exp.
    \end{equation}
    Wynika stąd, że $\exp$ jest klasy $C^\oo (\R)$.
\end{twier}

\begin{dow}{}
    Niech $t \in R$ t.ż. $t = \log x$. Wtedy z właności pochodnej funkcji odwrotnej: \begin{equation}
        \exp'(t) = \exp'(\log x) = \frac{1}{\log' x} = x = exp(t).
    \end{equation}
\end{dow}


\paragraph{} Zauważmy, że skoro $\log(1) = 0$, to $\exp(0) = \exp^{(n)}(0) = 1 \forall_{n\in \N}$. Możemy więc rozpisać wzór Taylora dla $\exp$ wokół zera: \begin{equation}
    \exp(x) = \sum_{n = 0}^{N - 1} \frac{1}{n!} x^n + \frac{1}{N!} \exp(\xi) x^N,
\end{equation} dla $\xi$ pomiędzy 0 i $x$. Zauważmy jednak, że wyraz resztowy dąży do zera wraz z $N$: \begin{equation}
    \frac{1}{N!} \exp(\xi) x^N \leq \frac{1}{N!} \exp(|x|) t^N \xrightarrow{N \to \oo} 0.
\end{equation} Zatem $\exp$ jest granicą (punktową! - zob. następna sekcja) swoich rozwinięć Taylora.

\begin{obs}{}
    \begin{equation}
        \exp(x) = \sum_{n = 0}^{\oo} \frac{x ^ n}{n!}.
    \end{equation}
\end{obs}

\begin{defr}{Liczba e}
    Liczbą $e$ Eulera nazywamy wielkość: \begin{equation}
        e = \exp (1) = \sum_{n = 0}^{\oo} \frac{1}{n!}.
    \end{equation}
\end{defr}

\begin{defr}{Potęgowanie}
    Niech $a > 0$ i $t \in \R$. Definiujemy: \begin{equation}
        a ^ t = \exp(t \log x).
    \end{equation}
\end{defr}
\paragraph{} Zauważmy, że definicja ta, dla $t \in \Q$ pokrywa się ze standardową definicją potęgi (na mocy Tw. \ref*{twier:log-potega}). Podadto, z ciągłości $\exp$ i $\log$ wynika, że $a^t$ jest ciągłą funkcją zarówno podstawy jak i wykładnika. 
\paragraph{} Zgodnie z powyższą definicją mamy: \begin{equation}
    e ^ x = \exp x
\end{equation} i zapisów tych używać będziemy zamiennie.

\paragraph{} Powiemy teraz trochę o rozszerzeniu $\exp$ na całą przestrzeń liczb całkowitych.

\begin{defr}{Zespolona funkcja wykładnicza}
    Funckję $\exp: \C \to \C$: \begin{equation}
        \exp(z) = \sum_{n = 0}^{\oo} \frac{z ^ n}{n!},
    \end{equation} będącą rozszerzeń poprzednio zdefiniowanej funkcji $\exp$ na dziedzinę liczb zespolnych, dalej nazywać będziemy funkcją wykładniczą.
\end{defr}
\paragraph{} Zauważmy, że z kryterium D'Alemberta wynika, że jest to szereg zbieżny bezwzględnie. Rozszerzając dziedzinę $\exp$, tracimy właności bijekcji.

\begin{twier}{}
    Dla zespolonej funkcji wykładniczej zachodzi wzór: \begin{equation}
        exp(z) \exp(w) \ = \exp(w + z),
    \end{equation} gdzie $w, z \in \C$.
\end{twier}

%TODO Tw. Martensa

\begin{dow}{}
    \begin{equation}
        exp(z) \exp(w) = (\sum_{n = 0}^{\oo} \frac{z ^ n}{n!}) ( \sum_{n = 0}^{\oo} \frac{w ^ n}{n!}) = \sum_{n = 0}^{\oo} \sum_{k = 0}^{n} \frac{z^k}{ k!} \frac{w ^{n - k}}{(n - k)!} = \sum_{n = 0}^{\oo} \sum_{k = 0}^{n} \binom{n}{k}\frac{z^k w ^ (n - k)}{n!} = \sum_{n = 0}^{\oo} \frac{(z + w) ^ n}{ n!} = \exp( z + w).
    \end{equation} Skorzystaliśmy tu z tzw. Twierdzenia Mertensa (nie załączonego w tym skrypcie), które mówi, że jeśli oba szeregi są zbieżne bewzględnie, to ich iloczyn jest równy ich iloczynowi Cauchy'ego.
\end{dow}


\begin{twier}{}
    Zespolona funkcja wykładnicza jest: \begin{itemize}
        \item ciągła,
        \item różniczkowalna w sęsie zespolonym i jej pochodna jest równa $\exp$. Oznacza to, że: \begin{equation}
            \exp'(z) = \lim_{\C \ni w \to 0} \frac{\exp(z + w) - \exp(z)}{w} = \exp(z).
        \end{equation} Powyższa granica, to po wszykich ciągach liczb zespolonych dążących do $z$.
    \end{itemize}
\end{twier}

%TODO DOWÓD


\newpage
\section*{Koniec}
\newpage
\tableofcontents
\end{document}
